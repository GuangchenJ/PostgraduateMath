\documentclass[lang=cn,newtx,10pt,scheme=chinese]{elegantbook}

\title{ElegantBook:优美的 \LaTeX{} 书籍模板}
\subtitle{Elegant\LaTeX{} 经典之作}

\author{Ethan Deng \& Liam Huang \& syvshc \& sikouhjw \& Osbert Wang}
\institute{Elegant\LaTeX{} Program}
\date{2022/12/31}
\version{4.5}
\bioinfo{自定义}{信息}

\extrainfo{注意:本模板自 2023 年 1 月 1 日开始,不再更新和维护!}

\setcounter{tocdepth}{3}

\logo{logo-blue.png}
\cover{cover.jpg}

% 本文档命令
\usepackage{array}
\usepackage{amsmath}   % 用于 \pmatrix 矩阵环境
\usepackage{tasks}     % 用于水平排列选项
\usepackage{xifthen}   % 用于处理 \problem 命令的可选参数
\newcommand{\ccr}[1]{\makecell{{\color{#1}\rule{1cm}{1cm}}}}

% 修改标题页的橙色带
\definecolor{customcolor}{RGB}{32,178,170}
\colorlet{coverlinecolor}{customcolor}
\usepackage{cprotect}

% 定义题目计数器
\newcounter{problemcounter}[section]

\renewcommand{\problem}[2][]{%
  % \ifnum... 判断语句
  % 检查 problemcounter 计数器的值是否 > 0
%   \ifnum\value{problemcounter}>0 
%     % 如果 > 0 (即, 这不是第一题), 
%     % 就在题目前插入 \vspace (大空白)
%     \par\vspace{5cm}\noindent 
%   \else
%     % 如果 = 0 (即, 这是第一题),
%     % 就在题目前插入 \medskip (一个小间距)
%     \par\medskip\noindent 
%   \fi
  \par\medskip\noindent 
  %
  % --- 下面的代码不变 ---
  \stepcounter{problemcounter}% 计数器加 1 (第一题变1, 第二题变2...)
  \arabic{problemcounter}.% 打印题号 1.
  \ifthenelse{\isempty{#1}}{}{ 【#1】}% 打印可选信息 (25-2)
  \space % 打印一个空格
  #2 % 打印题目正文
}

% (可选) 全局设置 tasks 环境的样式
% 这里我们设置为4列,标签格式为 A. B. C. ...
\settasks{
  counter-format = \Alph*., % 标签格式为 A. B. ...
  label-width = 2em,         % 标签宽度
  item-indent = 2em,         % 选项内容缩进
}

\addbibresource[location=local]{reference.bib} % 参考文献,不要删除

\begin{document}

\maketitle
\frontmatter

\tableofcontents

\mainmatter

\chapter{高数}

\section{极限}

\section{一元函数微分学}

\section{一元函数积分学}

\problem[李六-1-1]{当 $x \to 0$ 时, 无穷小量
  $a_1 = \int_x^{2\sin x} (e^{t^2} - 1) \, \mathrm{d}t$, 
  $a_2 = \int_x^{e^x - 1} \ln \cos t \, \mathrm{d}t$, 
  $a_3 = \int_{x^2}^x \frac{\tan^3 t}{t} \, \mathrm{d}t$ 
  关于 $x$ 的阶数分别为 ( \quad )}

\begin{tasks}(4)
  \task 2, 3, 4.
  \task 3, 3, 3.
  \task 3, 5, 3.
  \task 3, 4, 3.
\end{tasks}

\section{微分方程}

% --- 题目 ---
\problem[李六-1-3]{在 $Oxy$ 平面上, 光滑曲线 $L$ 过 $(1,0)$ 点, 并且曲线上任意一点 $P(x,y) (x \ne 0)$ 处的切线斜率与直线 $OP$ 的斜率之差等于 $ax (a>0$ 为常数$)$. 如果 $L$ 与直线 $y=ax$ 所围成的平面图形的面积为 8, 则 $a$ 的值为 ( \quad )}

\begin{tasks}(4)
  \task 2.
  \task 4.
  \task 6.
  \task 8.
\end{tasks}

% --- 题目 ---
\problem[李六-1-11]{设 $f(x)$ 是定义在 $(-\infty, +\infty)$ 上以 $2\pi$ 为周期的二阶可导函数, 且满足等式 $f(x) + 2f'(x+\pi) = \sin x$,则 $f(x) = \underline{\hspace{6em}}.$}

\section{多元函数微分学}

\section{多元函数积分学}

\section{无穷级数}

\chapter{线代}

% --- 题目 1 (多选题) ---
\problem[25-2]{下列矩阵中,可以经过若干初等行变换得到矩阵 $\begin{pmatrix} 1 & 1 & 0 & 1 \\ 0 & 0 & 1 & 2 \\ 0 & 0 & 0 & 0 \end{pmatrix}$ 的是 ( \quad )}

% 使用 tasks 环境来排版选项
\begin{tasks}(4) % (4) 表示强制分为4列
  \task $\begin{pmatrix} 1 & 1 & 0 & 1 \\ 1 & 2 & 1 & 3 \\ 2 & 3 & 1 & 4 \end{pmatrix}$
  \task $\begin{pmatrix} 1 & 1 & 0 & 1 \\ 1 & 1 & 2 & 5 \\ 1 & 1 & 1 & 3 \end{pmatrix}$
  \task $\begin{pmatrix} 1 & 0 & 0 & 1 \\ 0 & 1 & 0 & 3 \\ 0 & 1 & 0 & 0 \end{pmatrix}$
  \task $\begin{pmatrix} 1 & 1 & 2 & 3 \\ 1 & 2 & 2 & 3 \\ 2 & 3 & 4 & 6 \end{pmatrix}$
\end{tasks}


\chapter{概率}

\section{随机事件和概率}

\problem[李六-1-10]{一颗陨石等可能地坠落在区域 $A_1, A_2, A_3, A_4$ 后, 有关部门千方百计地要找到它. 根据现有的搜索条件, 如果陨石坠落在 $A_i$, 则在该区域被找到的概率是 $p_i$ (这里 $p_i$ 是由 $A_i$ 的地貌条件决定的; $i=1,2,3,4$). 现对 $A_1$ 搜索后没有发现这块陨石, 则陨石坠落在 $A_4$ 的概率为 ( \quad )}

\begin{tasks}(4)
  \task $\frac{1}{3}$.
  \task $\frac{1}{4}$.
  \task $\frac{1-p_1}{4-p_1}$.
  \task $\frac{1}{4-p_1}$.
\end{tasks}

\section{随机变量及其分布}

\problem[李六-1-8]{设连续型随机变量 $X$ 的分布函数为 $F(x)$, 且 $F(0)=0$. 则下列函数可作为分布函数的是 ( \quad )}

% 注意:由于选项内容较长, 这里使用 (2) 即分为 2 列
% 如果您坚持 4 列, 请使用 (4)
\begin{tasks}(2)
  \task $G_1(x) = \begin{cases} 1+F\left(\frac{1}{x}\right), & x > 1, \\ 0, & x \le 1. \end{cases}$
  \task $G_2(x) = \begin{cases} 1-F\left(\frac{1}{x}\right), & x > 1, \\ 0, & x \le 1. \end{cases}$
  \task $G_3(x) = \begin{cases} F(x)-F\left(\frac{1}{x}\right), & x > 1, \\ 0, & x \le 1. \end{cases}$
  \task $G_4(x) = \begin{cases} F(x)+F\left(\frac{1}{x}\right), & x > 1, \\ 0, & x \le 1. \end{cases}$
\end{tasks}

\section{多维随机变量及其分布}

\section{随机变量的数字特征}

\section{大数定律与中心极限定理}

\section{数理统计的基本概念}

\section{参数估计}

\nocite{*}

\printbibliography[heading=bibintoc, title=\ebibname]
\appendix
\chapter{答案}

\end{document}
