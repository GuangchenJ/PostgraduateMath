\documentclass[lang=cn,math=mtpro2,11pt,scheme=chinese]{elegantbook}

\title{考研数学真题错题整理}
\subtitle{参考 2026 版《考研数学这十年》}

\author{Guangchen Jiang}
% \institute{Elegant\LaTeX{} Program}
\date{2025/12/31}
\version{4.5}
% \bioinfo{自定义}{信息}

% \extrainfo{注意:本模板自 2023 年 1 月 1 日开始,不再更新和维护!}

\setcounter{tocdepth}{3}

\logo{logo-blue.png}
\cover{cover.jpg}

% 本文档命令
\usepackage{array}
\usepackage{amsmath}   % 用于 \pmatrix 矩阵环境
\usepackage{tasks}     % 用于水平排列选项
\usepackage{xifthen} 
\usepackage{pifont}
\usepackage{}
% \usepackage{ifthen}       % 确保 \ifthenelse 命令可用
% \usepackage{pifont}       % 确保 \ding 命令可用 (用于图标)
% \usepackage{tcolorbox}    % 确保 tcolorbox 可用
\usepackage{graphicx}
\usepackage{fontawesome5}  % 用于提供图标,例如 \faPencil
% 加载 tcolorbox 的 "skins" 和 "theorems" 库
\tcbuselibrary{skins, theorems}  % 用于处理 \problem 命令的可选参数
\newcommand{\ccr}[1]{\makecell{{\color{#1}\rule{1cm}{1cm}}}}

% 修改标题页的橙色带
\definecolor{customcolor}{RGB}{32,178,170}
\colorlet{coverlinecolor}{customcolor}
\usepackage{cprotect}

\makeatother % 恢复@的原始含义

% 定义题目计数器
\newcounter{problemcounter}[section]

\renewcommand{\problem}[2][]{
  \par\medskip\noindent 
  %
  % --- 下面的代码不变 ---
  \stepcounter{problemcounter}% 计数器加 1 (第一题变1, 第二题变2...)
  \arabic{problemcounter}.% 打印题号 1.
  \ifthenelse{\isempty{#1}}{}{ 【#1】}% 打印可选信息 (25-2)
  \space % 打印一个空格
  #2 % 打印题目正文
}
\renewcommand{\proofname}{答案}
\renewcommand{\notename}{注}

\newcommand{\probtagged}[2][]{%
  \par\noindent%
  % 左侧小图标
  \makebox[0pt][r]{%
    \scriptsize\color{orange!90}\faExclamationTriangle\quad%
  }%
  % 正文部分
  \stepcounter{problemcounter}%
  \arabic{problemcounter}.%
  \ifthenelse{\isempty{#1}}{}{ 【#1】}%
  \space #2%
  \par%
}

% (可选) 全局设置 tasks 环境的样式
% 这里我们设置为4列,标签格式为 A. B. C. ...
\settasks{
  counter-format = (\Alph*), % 标签格式为 A. B. ...
  label-width = 2em,         % 标签宽度
  item-indent = 3em,         % 选项内容缩进
}
\newcommand{\ansat}[1]{%
  \par\nobreak % <--- 关键解决方案:结束上文,并禁止在此处分页
  \vspace{0.3em} % 在 proof 环境前添加您想要的间距
  \begin{proof}
    P#1
  \end{proof}
  \vspace{5em} % <--- 修正:在 proof 环境后添加一些间距 (例如 0.5em)
}

\makeatletter
\newcommand{\rmnum}[1]{\romannumeral #1}
\newcommand{\Rmnum}[1]{\expandafter\@slowromancap\romannumeral #1@}
\makeatother

% \usepackage{setspace} 
% \setstretch{1.3}

\addbibresource[location=local]{reference.bib} % 参考文献,不要删除

\begin{document}

% \maketitle
% \frontmatter

% \tableofcontents

\mainmatter

\chapter*{线代错题(乱序版)}

% --- 题目 ---
\problem[P186-12 (00-1)]{某试验性生产线每年1月份进行熟练工与非熟练工的人数统计, 然后将$\displaystyle \frac{1}{6}$熟练工支援其他生产部门, 其缺额由招收新的非熟练工补齐. 新、老非熟练工经过培训及实践至年终考核有$\displaystyle \frac{2}{5}$成为熟练工. 设第$\displaystyle n$年1月份统计的熟练工和非熟练工所占百分比分别为$\displaystyle x_n$和$\displaystyle y_n$, 记成向量$\displaystyle \begin{pmatrix} x_n \\ y_n \end{pmatrix}$.
\begin{enumerate}
\item[(1)] 求$\displaystyle \begin{pmatrix} x_{n+1} \\ y_{n+1} \end{pmatrix}$与$\begin{pmatrix} x_n \\ y_n \end{pmatrix}$的关系式并写成矩阵形式:
$$\begin{pmatrix} x_{n+1} \\ y_{n+1} \end{pmatrix} = \bm{A}\begin{pmatrix} x_n \\ y_n \end{pmatrix};$$
\item[(2)] 验证$\displaystyle \bm{\eta}_1=\begin{pmatrix} 4 \\ 1 \end{pmatrix}$, $\displaystyle \bm{\eta}_2=\begin{pmatrix} -1 \\ 1 \end{pmatrix}$是$\bm{A}$的两个线性无关的特征向量,并求出相应的特征值;
\item[(3)] 当$\displaystyle \begin{pmatrix} x_1 \\ y_1 \end{pmatrix} = \begin{pmatrix} \frac{1}{2} \\ \frac{1}{2} \end{pmatrix}$时, 求$\displaystyle \begin{pmatrix} x_{n+1} \\ y_{n+1} \end{pmatrix}$.
\end{enumerate}}
\ansat{346;【真题精选】 - 考点:矩阵的相似和相似对角化 - 12}

% --- 题目 ---
\problem[P179-3 (02-3)]{设$\displaystyle \bm{A}$是$\displaystyle n$阶实对称矩阵, $\displaystyle \bm{P}$是$\displaystyle n$阶可逆矩阵, 已知$\displaystyle n$维列向量$\displaystyle \bm{\alpha}$是$\displaystyle \bm{A}$的属于特征值$\displaystyle \lambda$的特征向量, 则矩阵$\displaystyle \left(\bm{P}^{-1}\bm{A}\bm{P}\right)^{\mathrm{T}}$属于特征值$\displaystyle \lambda$的特征向量是( \quad )
\begin{tasks}(2)
  \task $\displaystyle \bm{P}^{-1}\bm{\alpha}$.
  \task $\displaystyle \bm{P}^{\mathrm{T}}\bm{\alpha}$.
  \task $\displaystyle \bm{P}\bm{\alpha}$.
  \task $\displaystyle \left(\bm{P}^{-1}\right)^{\mathrm{T}}\bm{\alpha}$.
\end{tasks}}
\ansat{341;【真题精选】 - 考点:矩阵的特征值和特征向量 - 3}

% --- 题目 ---
\problem[P179-变式 1.2]{设3阶实对称矩阵$\displaystyle \bm{A}$的秩为2,$\displaystyle \bm{A}^2=\bm{A}$,且$\bm{A}(1,-1,1)^{\mathrm{T}}=\bm{0}$,求$\bm{A}$的特征值与特征向量.}
\ansat{341;【方法探究】 - 考点:矩阵的特征值和特征向量 - 变式 1.2}
\vspace{4em}

% --- 题目 ---
\problem[P193-8 (13-1,2,3)]{设二次型
$$f(x_1,x_2,x_3)=2 \left(a_1x_1+a_2x_2+a_3x_3\right)^2 + \left(b_1x_1+b_2x_2+b_3x_3\right)^2,$$
记$\displaystyle \bm{\alpha}=\begin{pmatrix} a_1 \\ a_2 \\ a_3 \end{pmatrix}$, $\displaystyle \bm{\beta}=\begin{pmatrix} b_1 \\ b_2 \\ b_3 \end{pmatrix}$.
\begin{enumerate}
\item[(1)] 证明二次型$\displaystyle f$对应的矩阵为$\displaystyle 2\bm{\alpha}\bm{\alpha}^{\mathrm{T}}+\bm{\beta}\bm{\beta}^{\mathrm{T}}$;
\item[(2)] 若$\displaystyle \bm{\alpha}, \bm{\beta}$正交且均为单位向量, 证明二次型$\displaystyle f$在正交变换下的标准形为$\displaystyle 2y_1^2+y_2^2$.
\end{enumerate}}
\begin{note}
  主要错的是 (2)
\end{note}
\ansat{352;【真题精选】 - 考点一:化二次型为标准型 - 8}

% --- 题目 ---
\problem[P173-4 (03-1)]{设有齐次线性方程组 $\displaystyle \bm{A}\bm{x}=\bm{0}$ 和 $\displaystyle \bm{B}\bm{x}=\bm{0}$, 其中 $\displaystyle \bm{A}, \bm{B}$ 均为 $\displaystyle m \times n$ 矩阵, 现有 4 个命题:
\\
\textcircled{1} 若 $\displaystyle \bm{A}\bm{x} =\bm{0}$ 的解均是 $\displaystyle \bm{B}\bm{x}=\bm{0}$ 的解, 则 $\displaystyle r(\bm{A}) \ge r(\bm{B})$;
\\
\textcircled{2} 若 $\displaystyle r(\bm{A}) \ge r(\bm{B})$, 则 $\displaystyle \bm{A}\bm{x}=\bm{0}$ 的解均是 $\displaystyle \bm{B}\bm{x}=\bm{0}$ 的解;
\\
\textcircled{3} 若 $\displaystyle \bm{A}\bm{x}=\bm{0}$ 与 $\displaystyle \bm{B}\bm{x}=\bm{0}$ 同解, 则 $\displaystyle r(\bm{A}) = r(\bm{B})$;
\\
\textcircled{4} 若 $\displaystyle r(\bm{A}) = r(\bm{B})$, 则 $\displaystyle \bm{A}\bm{x}=\bm{0}$ 与 $\displaystyle \bm{B}\bm{x}=\bm{0}$ 同解.
\\
以上命题中正确的是~(~\quad~)}
\begin{tasks}(4)
  \task \textcircled{1} \textcircled{2}.
  \task \textcircled{1} \textcircled{3}.
  \task \textcircled{2} \textcircled{4}.
  \task \textcircled{3} \textcircled{4}.
\end{tasks}
\ansat{338;【真题精选】 - 考点:线性方程组的解的结构 - 4}

% --- 题目 ---
\problem[P156-5 (08-1)]{设 $\displaystyle \boldsymbol{\alpha}, \boldsymbol{\beta}$ 为 3 维列向量, 矩阵 $\displaystyle \bm{A} = \boldsymbol{\alpha}\boldsymbol{\alpha}^{\mathrm{T}} + \boldsymbol{\beta}\boldsymbol{\beta}^{\mathrm{T}}$, 其中 $\displaystyle \boldsymbol{\alpha}^{\mathrm{T}}$ 为 $\displaystyle \boldsymbol{\alpha}$ 的转置, $\displaystyle \boldsymbol{\beta}^{\mathrm{T}}$ 为 $\displaystyle \boldsymbol{\beta}$ 的转置. 证明:}
\begin{enumerate}
\item[(1)] $\displaystyle r(\bm{A}) \le 2$;
\item[(2)] 若 $\displaystyle \boldsymbol{\alpha}, \boldsymbol{\beta}$ 线性相关, 则 $\displaystyle r(\bm{A}) < 2$.
\end{enumerate}
\ansat{327;【真题精选】 - 考点三:矩阵的秩与等价 - 5}
\vspace{8em}

% --- 题目 ---
\problem[P170-4 (21-2)]{设 3 阶矩阵 $\displaystyle \bm{A}=(\bm{\alpha}_1, \bm{\alpha}_2, \bm{\alpha}_3), \ \bm{B}=(\bm{\beta}_1, \bm{\beta}_2, \bm{\beta}_3)$. 若向量组 $\displaystyle \bm{\alpha}_1, \bm{\alpha}_2, \bm{\alpha}_3$ 可以由向量组 $\displaystyle \bm{\beta}_1, \bm{\beta}_2, \bm{\beta}_3$ 线性表出, 则~(~\quad~)}
\begin{tasks}(2)
  \task $\displaystyle \bm{A}\bm{x}=\bm{0}$ 的解均为 $\displaystyle \bm{Bx}=\bm{0}$ 的解.
  \task $\displaystyle \bm{A}^{\mathrm{T}}\bm{x}=\bm{0}$ 的解均为 $\displaystyle \bm{B}^{\mathrm{T}}\bm{x}=\bm{0}$ 的解.
  \task $\displaystyle \bm{B}\bm{x}=\bm{0}$ 的解均为 $\displaystyle \bm{A}\bm{x}=\bm{0}$ 的解.
  \task $\displaystyle \bm{B}^{\mathrm{T}}\bm{x}=\bm{0}$ 的解均为 $\displaystyle \bm{A}^{\mathrm{T}}\bm{x}=\bm{0}$ 的解.
\end{tasks}
\ansat{336;【十年真题】 - 考点:线性方程组的解的结构 - 4}

% --- 题目 ---
\problem[P187-15 (92-4)]{设矩阵$\displaystyle \bm{A}, \bm{B}$相似, 其中$\displaystyle \bm{A}=\begin{pmatrix} -2 & 0 & 0 \\ 2 & x & 2 \\ 3 & 1 & 1 \end{pmatrix}$, $\displaystyle \bm{B}=\begin{pmatrix} -1 & 0 & 0 \\ 0 & 2 & 0 \\ 0 & 0 & y \end{pmatrix}$.
\begin{enumerate}
\item[(1)] 求$\displaystyle x, y$的值;
\item[(2)] 求可逆矩阵$\displaystyle \bm{P}$, 使$\displaystyle \bm{P}^{-1}\bm{A}\bm{P}=\bm{B}$.
\end{enumerate}}
\ansat{346;【真题精选】 - 考点:矩阵的相似和相似对角化 - 12}

% --- 题目 ---
\problem[P150-1 (25-2)]{下列矩阵中, 可以经过若干初等行变换得到矩阵 $\displaystyle \begin{pmatrix} 1 & 1 & 0 & 1 \\ 0 & 0 & 1 & 2 \\ 0 & 0 & 0 & 0 \end{pmatrix}$ 的是~(~\quad~)}
\begin{tasks}(2)
  \task $\displaystyle \begin{pmatrix} 1 & 1 & 0 & 1 \\ 1 & 2 & 1 & 3 \\ 2 & 3 & 1 & 4 \end{pmatrix}.$
  \task $\displaystyle \begin{pmatrix} 1 & 1 & 0 & 1 \\ 1 & 1 & 2 & 5 \\ 1 & 1 & 1 & 3 \end{pmatrix}.$
  \task $\displaystyle \begin{pmatrix} 1 & 0 & 0 & 1 \\ 0 & 1 & 0 & 3 \\ 0 & 1 & 0 & 0 \end{pmatrix}.$
  \task $\displaystyle \begin{pmatrix} 1 & 1 & 2 & 3 \\ 1 & 2 & 2 & 3 \\ 2 & 3 & 4 & 6 \end{pmatrix}.$
\end{tasks}
\ansat{325;【十年真题】 - 考点二:矩阵的初等变换与初等矩阵 - 1}
\vspace{10em}

% --- 题目 ---
\problem[P180-8 (17-1,2,3)]{已知矩阵$\displaystyle \bm{A}=\begin{pmatrix} 2 & 0 & 0 \\ 0 & 2 & 1 \\ 0 & 0 & 1 \end{pmatrix}$, $\displaystyle \bm{B}=\begin{pmatrix} 2 & 1 & 0 \\ 0 & 2 & 0 \\ 0 & 0 & 1 \end{pmatrix}$, $\displaystyle \bm{C}=\begin{pmatrix} 1 & 0 & 0 \\ 0 & 2 & 0 \\ 0 & 0 & 2 \end{pmatrix}$,则( \quad )
\begin{tasks}(2)
  \task $\bm{A}$与$\bm{C}$相似, $\bm{B}$与$\bm{C}$相似.
  \task $\bm{A}$与$\bm{C}$相似, $\bm{B}$与$\bm{C}$不相似.
  \task $\bm{A}$与$\bm{C}$不相似, $\bm{B}$与$\bm{C}$相似.
  \task $\bm{A}$与$\bm{C}$不相似, $\bm{B}$与$\bm{C}$不相似.
\end{tasks}}
\ansat{342;【十年真题】 - 考点:矩阵的相似和相似对角化 - 8}

% --- 题目 ---
\problem[P149-4 (96-5)]{5 阶行列式 $\displaystyle \begin{vmatrix} 1-a & a & 0 & 0 & 0 \\ -1 & 1-a & a & 0 & 0 \\ 0 & -1 & 1-a & a & 0 \\ 0 & 0 & -1 & 1-a & a \\ 0 & 0 & 0 & -1 & 1-a \end{vmatrix} = \underline{\hspace{4em}}.$}
\ansat{324;【真题精选】 - 考点:具体行列式的计算 - 4}

% --- 题目 ---
\problem[P173-8 (98-1)]{已知线性方程组
$$
\begin{cases}
a_{11}x_1 + a_{12}x_2 + \cdots + a_{1,2n}x_{2n} = 0, \\
a_{21}x_1 + a_{22}x_2 + \cdots + a_{2,2n}x_{2n} = 0, \\
\quad \cdots\cdots\cdots \\
a_{n1}x_1 + a_{n2}x_2 + \cdots + a_{n,2n}x_{2n} = 0
\end{cases}
$$
的一个基础解系为 $\displaystyle \left(b_{11}, b_{12}, \cdots, b_{1,2n}\right)^{\mathrm{T}}, \left(b_{21}, b_{22}, \cdots, b_{2,2n}\right)^{\mathrm{T}}, \cdots, \left(b_{n1}, b_{n2}, \cdots, b_{n,2n}\right)^{\mathrm{T}}$, 则线性方程组
$$
\begin{cases}
b_{11}y_1 + b_{12}y_2 + \cdots + b_{1,2n}y_{2n} = 0, \\
b_{21}y_1 + b_{22}y_2 + \cdots + b_{2,2n}y_{2n} = 0, \\
\quad \cdots\cdots\cdots \\
b_{n1}y_1 + b_{n2}y_2 + \cdots + b_{n,2n}y_{2n} = 0
\end{cases}
$$
的通解为 \underline{\hspace{4em}}.}
\ansat{338;【真题精选】 - 考点:线性方程组的解的结构 - 8}

% --- 题目 ---
\problem[P173-10 (05-1,2)]{已知3阶矩阵$\displaystyle \bm{A}$的第一行是$\displaystyle (a,b,c)$, $\displaystyle a,b,c$不全为零,矩阵 $\displaystyle \bm{B}=\begin{pmatrix} 1 & 2 & 3 \\ 2 & 4 & 6 \\ 3 & 6 & k \end{pmatrix}$ ($\displaystyle k$为常数), 且$\displaystyle \bm{AB}=\bm{O}$, 求线性方程组$\displaystyle \bm{A}\bm{x}=\bm{0}$的通解.}
\ansat{338;【真题精选】 - 考点:线性方程组的解的结构 - 10}

% --- 题目 ---
\problem[P190-2 (21-1)]{设矩阵$\displaystyle \bm{A}=\begin{pmatrix} a & 1 & -1 \\ 1 & a & -1 \\ -1 & -1 & a \end{pmatrix}$.
\begin{enumerate}
\item[(1)] 求正交矩阵$\displaystyle \bm{P}$, 使$\displaystyle \bm{P}^{\mathrm{T}}\bm{A}\bm{P}$为对角矩阵;
\item[(2)] 求正定矩阵$\displaystyle \bm{C}$, 使$\displaystyle \bm{C}^2=(a+3)\bm{E}-\bm{A}$, 其中$\bm{E}$为3阶单位矩阵.
\end{enumerate}}
\begin{note}
  主要错的是 (2)
\end{note}
\ansat{350;【十年真题】 - 考点三:正定二次型与正定矩阵 - 2}

% --- 题目 ---
\problem[P156-3 (01-3)]{设}
\[ \bm{A} = \begin{pmatrix} a_{11} & a_{12} & a_{13} & a_{14} \\ a_{21} & a_{22} & a_{23} & a_{24} \\ a_{31} & a_{32} & a_{33} & a_{34} \\ a_{41} & a_{42} & a_{43} & a_{44} \end{pmatrix}, \bm{B} = \begin{pmatrix} a_{14} & a_{13} & a_{12} & a_{11} \\ a_{24} & a_{23} & a_{22} & a_{21} \\ a_{34} & a_{33} & a_{32} & a_{31} \\ a_{44} & a_{43} & a_{42} & a_{41} \end{pmatrix}, \]
\[ \bm{P}_1 = \begin{pmatrix} 0 & 0 & 0 & 1 \\ 0 & 1 & 0 & 0 \\ 0 & 0 & 1 & 0 \\ 1 & 0 & 0 & 0 \end{pmatrix}, \bm{P}_2 = \begin{pmatrix} 1 & 0 & 0 & 0 \\ 0 & 0 & 1 & 0 \\ 0 & 1 & 0 & 0 \\ 0 & 0 & 0 & 1 \end{pmatrix}, \]
$\displaystyle \bm{A}$ 可逆, 则 $\displaystyle \bm{B}^{-1} =~(~\quad~)$
\begin{tasks}(2)
  \task $\displaystyle \bm{A}^{-1}\bm{P}_1\bm{P}_2.$
  \task $\displaystyle \bm{P}_1\bm{A}^{-1}\bm{P}_2.$
  \task $\displaystyle \bm{P}_1\bm{P}_2\bm{A}^{-1}.$
  \task $\displaystyle \bm{P}_2\bm{A}^{-1}\bm{P}_1.$
\end{tasks}
\ansat{327;【真题精选】 - 考点二:矩阵的初等变换与初等矩阵 - 3}

% --- 题目 ---
\problem[P172-变式 (07-1,2,3)]{设线性方程组
\[ \begin{cases} x_1 + x_2 + x_3 = 0, \\ x_1 + 2x_2 + ax_3 = 0, \\ x_1 + 4x_2 + a^2 x_3 = 0 \end{cases} \]
与方程 $$x_1 + 2x_2 + x_3 = a - 1$$ 有公共解, 求 $\displaystyle a$ 的值及所有公共解.}
\ansat{337;【方法探究】 - 考点:线性方程组的解的结构 - 变式}

% --- 题目 ---
\problem[P162-8 (00-2)]{设 $\displaystyle \bm{\alpha} = \begin{pmatrix} 1 \\ 2 \\ 1 \end{pmatrix}, \ \bm{\beta} = \begin{pmatrix} 1 \\ \frac{1}{2} \\ 0 \end{pmatrix}, \ \bm{\gamma} = \begin{pmatrix} 0 \\ 0 \\ 8 \end{pmatrix}, \bm{A} = \bm{\alpha}\bm{\beta}^{\mathrm{T}}, \bm{B} = \bm{\beta}^{\mathrm{T}}\bm{\alpha}.$
\\
其中 $\displaystyle \bm{\beta}^{\mathrm{T}}$ 是 $\displaystyle \bm{\beta}$ 的转置, 求解方程
\\
$$2\bm{B}^2 \bm{A}^2 \bm{x} = \bm{A}^4 \bm{x} + \bm{B}^4 \bm{x} + \bm{\gamma}.$$
}
\ansat{331;【真题精选】 - 考点一:线性方程组的解的情况及求解 - 8}

% --- 题目 ---
\problem[P180-5 (22-2,3)]{设$\displaystyle \bm{A}$为3阶矩阵, $\displaystyle \bm{\Lambda}=\begin{pmatrix} 1 & 0 & 0 \\ 0 & -1 & 0 \\ 0 & 0 & 0 \end{pmatrix}$, 则$\displaystyle \bm{A}$的特征值为 $\displaystyle 1,-1,0$ 的充分必要条件是( \quad )
\begin{tasks}(2)
  \task 存在可逆矩阵$\displaystyle \bm{P},\bm{Q}$, 使得$\displaystyle \bm{A}=\bm{P}\bm{\Lambda}\bm{Q}$.
  \task 存在可逆矩阵$\displaystyle \bm{P}$, 使得$\displaystyle  \bm{A}=\bm{P}\bm{\Lambda}\bm{P}^{-1}$.
  \task 存在正交矩阵$\displaystyle \bm{Q}$, 使得$\displaystyle \bm{A}=\bm{Q}\bm{\Lambda}\bm{Q}^{-1}$.
  \task 存在可逆矩阵$\displaystyle \bm{P}$, 使得$\displaystyle \bm{A}=\bm{P}\bm{\Lambda}\bm{P}^{\mathrm{T}}$.
\end{tasks}}
\ansat{341;【十年真题】 - 考点:矩阵的相似和相似对角化 - 5}

% --- 题目 ---
\problem[P150-4 (22-1)]{已知矩阵 $\displaystyle \bm{A}$ 和 $\displaystyle \bm{E}-\bm{A}$ 可逆, 其中 $\displaystyle \bm{E}$ 为单位矩阵. 若矩阵 $\displaystyle \bm{B}$ 满足 $\displaystyle [\bm{E} - (\bm{E} - \bm{A})^{-1}]\bm{B} = \bm{A}$, 则 $\displaystyle \bm{B} - \bm{A} = \underline{\hspace{4em}}. $}
\ansat{325;【十年真题】 - 考点一:矩阵的运算 - 4}

% --- 题目 ---
\problem[P171-9 (25-2)]{设矩阵 $\displaystyle \bm{A} = (\bm{\alpha}_1, \bm{\alpha}_2, \bm{\alpha}_3, \bm{\alpha}_4)$. 若 $\displaystyle \bm{\alpha}_1, \bm{\alpha}_2, \bm{\alpha}_3$ 线性无关, 且 $\displaystyle \bm{\alpha}_1 + \bm{\alpha}_2 = \bm{\alpha}_3 + \bm{\alpha}_4$, 则方程组 $\displaystyle \bm{A}\bm{x} = \bm{\alpha}_1 + 4\bm{\alpha}_4$ 的通解为 $\displaystyle \bm{x} = \underline{\hspace{4em}}.$}
\ansat{337;【十年真题】 - 考点:线性方程组的解的结构 - 9}

% --- 题目 ---
\problem[P193 (01-3)]{设$\displaystyle \bm{A}$为$\displaystyle n$阶实对称矩阵, $\displaystyle r\left(\bm{A}\right)=n$, $\displaystyle A_{ij}$是$\displaystyle \bm{A}=\left(a_{ij}\right)_{n \times n}$中元素$\displaystyle a_{ij}$的代数余子式 $\displaystyle (i,j=1,2,\cdots,n)$, 二次型$\displaystyle f(x_1,x_2,\cdots,x_n)=\sum_{i=1}^n \sum_{j=1}^n \frac{A_{ij}}{|\bm{A}|}x_i x_j$.
\begin{enumerate}
\item[(1)] 记$\displaystyle \bm{x}=(x_1,x_2,\cdots,x_n)^{\mathrm{T}}$, 把$\displaystyle f(x_1,x_2,\cdots,x_n)$写成矩阵形式, 并证明二次型$\displaystyle f(\bm{x})$的矩阵为$\displaystyle \bm{A}^{-1}$;
\item[(2)] 二次型$\displaystyle g(\bm{x})=\bm{x}^{\mathrm{T}}\bm{A}\bm{x}$与$\displaystyle f(\bm{x})$的规范形是否相同? 说明理由.
\end{enumerate}}
\ansat{352;【真题精选】 - 考点二:矩阵的合同}

% --- 题目 ---
\problem[P162-2 (14-1,2,3)]{设矩阵 $\displaystyle \bm{A} = \begin{pmatrix} 1 & -2 & 3 & -4 \\ 0 & 1 & -1 & 1 \\ 1 & 2 & 0 & -3 \end{pmatrix}, \bm{E}$ 为 3 阶单位矩阵.}
\begin{enumerate}
\item[(1)] 求方程组 $\displaystyle \bm{Ax} = \bm{0}$ 的一个基础解系;
\item[(2)] 求满足 $\displaystyle \bm{AB} = \bm{E}$ 的所有矩阵 $\displaystyle \bm{B}$.
\end{enumerate}
\ansat{332;【真题精选】 - 考点二:矩阵方程组的解的情况及求解 - 2}

% --- 题目 ---
\problem[P158-5 (23-2,3)]{已知线性方程组}
\[ \begin{cases} ax_1 + x_3 = 1, \\ x_1 + ax_2 + x_3 = 0, \\ x_1 + 2x_2 + ax_3 = 0, \\ ax_1 + bx_2 = 2 \end{cases} \]
{有解, 其中 $\displaystyle a, b$ 为常数. 若 $\displaystyle \begin{vmatrix} a & 0 & 1 \\ 1 & a & 1 \\ 1 & 2 & a \end{vmatrix} = 4$, 则 $\displaystyle \begin{vmatrix} 1 & a & 1 \\ 1 & 2 & a \\ a & b & 0 \end{vmatrix} = \underline{\hspace{4em}}. $}
\ansat{328;【十年真题】 - 考点一:线性方程组的解的情况及求解 - 5}
\vspace{2em}

% --- 题目 ---
\problem[P185-7 (07-1,2,3)]{设3阶实对称矩阵$\displaystyle \bm{A}$的特征值$\lambda_1=1$, $\lambda_2=2$, $\lambda_3=-2$,且$\displaystyle \bm{\alpha}_1=(1,-1,1)^{\mathrm{T}}$是$\displaystyle \bm{A}$的属于$\displaystyle \lambda_1$的一个特征向量. 记$\displaystyle \bm{B}=\bm{A}^5-4\bm{A}^3+\bm{E}$, 其中$\bm{E}$为3阶单位矩阵.
\begin{enumerate}
\item[(1)] 验证$\displaystyle \bm{\alpha}_1$是矩阵$\displaystyle \bm{B}$的特征向量, 并求$\displaystyle \bm{B}$的全部特征值与特征向量;
\item[(2)] 求矩阵 $\displaystyle \bm{B}$ .
\end{enumerate}}
\ansat{345;【真题精选】 - 考点:矩阵的相似和相似对角化 - 7}

% --- 题目 ---
\problem[P148-例2 (1)]{$\displaystyle n$ 阶行列式 $\displaystyle \begin{vmatrix} 1 & -1 & & & & \\ 2 & a & -1 & & & \\ 3 & & a & -1 & & \\ \vdots & & \ddots & \ddots & & \\ n-1 & & & & a & -1 \\ n & & & & & a \end{vmatrix} = \underline{\hspace{4em}}.$}
\ansat{148;【方法探究】 - 考点:具体行列式的计算 - 例2 (1)}

% --- 题目 ---
\problem[P162-1 (15-2,3)]{设矩阵 $\displaystyle \bm{A} = \begin{pmatrix} a & 1 & 0 \\ 1 & a & -1 \\ 0 & 1 & a \end{pmatrix}$, 且 $\displaystyle \bm{A}^3 = \bm{O}$.}
\begin{enumerate}
\item[(1)] 求 $\displaystyle a$ 的值;
\item[(2)] 若矩阵 $\displaystyle \bm{X}$ 满足 $\displaystyle \bm{X} - \bm{XA}^2 - \bm{AX} + \bm{AXA}^2 = \bm{E}$, 其中 $\displaystyle \bm{E}$ 为 3 阶单位矩阵, 求 $\displaystyle \bm{X}$.
\end{enumerate}
\ansat{332;【真题精选】 - 考点二:矩阵方程组的解的情况及求解 - 1}

% --- 题目 ---
\problem[P188-2 (25-2)]{设矩阵$\displaystyle \begin{pmatrix} 1 & 2 & 0 \\ 2 & a & 0 \\ 0 & 0 & b \end{pmatrix}$有一个正特征值和两个负特征值, 则( \quad )
\begin{tasks}(2)
  \task $a>4, b>0$.
  \task $a<4, b>0$.
  \task $a>4, b<0$.
  \task $a<4, b<0$.
\end{tasks}}
\ansat{347;【十年真题】 - 考点一:化二次型为标准型 - 2}

% --- 题目 ---
\problem[P150-3 (24-1)]{设实矩阵 $\displaystyle \bm{A} = \begin{pmatrix} a+1 & a \\ a & a \end{pmatrix}$. 若对任意实向量 $\displaystyle \boldsymbol{\alpha} = \begin{pmatrix} x_1 \\ x_2 \end{pmatrix}, \ \boldsymbol{\beta} = \begin{pmatrix} y_1 \\ y_2 \end{pmatrix}$, 
$$\displaystyle (\boldsymbol{\alpha}^{\mathrm{T}} \bm{A} \boldsymbol{\beta})^2 \le \boldsymbol{\alpha}^{\mathrm{T}} \bm{A} \boldsymbol{\alpha} \cdot \boldsymbol{\beta}^{\mathrm{T}} \bm{A} \boldsymbol{\beta}$$ 
都成立, 则 $\displaystyle a$ 的取值范围为 \underline{\hspace{4em}}.}
\ansat{325;【十年真题】 - 考点一:矩阵的运算 - 3}

% --- 题目 ---
\problem[P189-2 (23-1)]{已知二次型
$$f(x_1,x_2,x_3)=x_1^2+2x_2^2+2x_3^2+2x_1x_2-2x_1x_3,$$
$$g(y_1,y_2,y_3)=y_1^2+y_2^2+y_3^2+2y_2y_3.$$
\begin{enumerate}
\item[(1)] 求可逆变换$\displaystyle \bm{x}=\bm{P}\bm{y}$将$\displaystyle f(x_1,x_2,x_3)$化为$\displaystyle g(y_1,y_2,y_3)$;
\item[(2)] 是否存在正交变换$\displaystyle \bm{x}=\bm{Q}\bm{y}$将$\displaystyle f(x_1,x_2,x_3)$化为$\displaystyle g(y_1,y_2,y_3)$?
\end{enumerate}}
\ansat{349;【十年真题】 - 考点二:矩阵的合同 - 2}

% --- 题目 ---
\problem[P168-4 (06-1,2,3)]{设 $\displaystyle \bm{\alpha}_1, \bm{\alpha}_2, \dots, \bm{\alpha}_s$ 均为 $\displaystyle n$ 维列向量, $\displaystyle \bm{A}$ 是 $\displaystyle m \times n$ 矩阵, 下列选项正确的是~(~\quad~)}
\begin{tasks}(1)
  \task 若 $\displaystyle \bm{\alpha}_1, \bm{\alpha}_2, \dots, \bm{\alpha}_s$ 线性相关, 则 $\displaystyle \bm{A}\bm{\alpha}_1, \bm{A}\bm{\alpha}_2, \dots, \bm{A}\bm{\alpha}_s$ 线性相关.
  \task 若 $\displaystyle \bm{\alpha}_1, \bm{\alpha}_2, \dots, \bm{\alpha}_s$ 线性相关, 则 $\displaystyle \bm{A}\bm{\alpha}_1, \bm{A}\bm{\alpha}_2, \dots, \bm{A}\bm{\alpha}_s$ 线性无关.
  \task 若 $\displaystyle \bm{\alpha}_1, \bm{\alpha}_2, \dots, \bm{\alpha}_s$ 线性无关, 则 $\displaystyle \bm{A}\bm{\alpha}_1, \bm{A}\bm{\alpha}_2, \dots, \bm{A}\bm{\alpha}_s$ 线性相关.
  \task 若 $\displaystyle \bm{\alpha}_1, \bm{\alpha}_2, \dots, \bm{\alpha}_s$ 线性无关, 则 $\displaystyle \bm{A}\bm{\alpha}_1, \bm{A}\bm{\alpha}_2, \dots, \bm{A}\bm{\alpha}_s$ 线性无关.
\end{tasks}
\ansat{335;【真题精选】 - 考点:向量组的线性相关性、线性表示及秩 - 4}

% --- 题目 ---
\problem[P181-15 (20-1,2,3)]{设$\displaystyle \bm{A}$为2阶矩阵, $\displaystyle \bm{P}=\left(\bm{\alpha},\bm{A}\bm{\alpha}\right)$, 其中$\displaystyle \bm{\alpha}$是非零向量且不是$\displaystyle \bm{A}$的特征向量.
\begin{enumerate}
\item[(1)] 证明$\displaystyle \bm{P}$为可逆矩阵;
\item[(2)] 若$\displaystyle \bm{A}^2\bm{\alpha}+\bm{A}\bm{\alpha}-6\bm{\alpha}=\bm{0}$, 求$\displaystyle \bm{P}^{-1}\bm{A}\bm{P}$, 并判断$\displaystyle \bm{A}$是否相似于对角矩阵.
\end{enumerate}}
\ansat{343;【十年真题】 - 考点:矩阵的相似和相似对角化 - 15}
\vspace{5em}

% --- 题目 ---
\problem[P174-12 (94-1)]{设四元线性齐次方程组(\Rmnum{1})为$\displaystyle \begin{cases} x_1 + x_2 = 0, \\ x_2 - x_4 = 0. \end{cases}$又已知某线性齐次方程组(\Rmnum{2})的通解为$k_1(0,1,1,0)^{\mathrm{T}} + k_2(-1,2,2,1)^{\mathrm{T}}$.
\begin{enumerate}
\item[(1)] 求线性方程组(\Rmnum{1})的基础解系;
\item[(2)] 问线性方程组(\Rmnum{1})和(\Rmnum{2})是否有非零公共解? 若有,则求出所有的非零公共解. 若没有,则说明理由.
\end{enumerate}}
\ansat{339;【真题精选】 - 考点:线性方程组的解的结构 - 12}

% --- 题目 ---
\problem[P179-1 (08-1,2,3)]{设$\displaystyle \bm{A}$为$\displaystyle n$阶非零矩阵, $\displaystyle \bm{E}$为$\displaystyle n$阶单位矩阵. 若$\displaystyle \bm{A}^3=\bm{O}$,则( \quad )
\begin{tasks}(2)
  \task $\displaystyle \bm{E}-\bm{A}$不可逆, $\displaystyle \bm{E}+\bm{A}$不可逆.
  \task $\displaystyle \bm{E}-\bm{A}$不可逆, $\displaystyle \bm{E}+\bm{A}$可逆.
  \task $\displaystyle \bm{E}-\bm{A}$可逆, $\displaystyle \bm{E}+\bm{A}$可逆.
  \task $\displaystyle \bm{E}-\bm{A}$可逆, $\displaystyle \bm{E}+\bm{A}$不可逆.
\end{tasks}}
\ansat{341;【真题精选】 - 考点:矩阵的特征值和特征向量 - 1}

% --- 题目 ---
\problem[P170-5 (21-3)]{设 $\displaystyle \bm{A} = (\bm{\alpha}_1, \bm{\alpha}_2, \bm{\alpha}_3, \bm{\alpha}_4)$ 为 4 阶正交矩阵. 若矩阵 $\displaystyle \bm{B} = \begin{pmatrix} \bm{\alpha}_1^{\mathrm{T}} \\ \bm{\alpha}_2^{\mathrm{T}} \\ \bm{\alpha}_3^{\mathrm{T}} \end{pmatrix}, \ \bm{\beta} = \begin{pmatrix} 1 \\ 1 \\ 1 \end{pmatrix}, k$ 表示任意常数, 则线性方程组 $\displaystyle \bm{B}\bm{x} = \bm{\beta}$ 的通解 $\displaystyle \bm{x} =~(~\quad~)$
}
\begin{tasks}(2)
  \task $\displaystyle \bm{\alpha}_2 + \bm{\alpha}_3 + \bm{\alpha}_4 + k\bm{\alpha}_1.$
  \task $\displaystyle \bm{\alpha}_1 + \bm{\alpha}_3 + \bm{\alpha}_4 + k\bm{\alpha}_2.$
  \task $\displaystyle \bm{\alpha}_1 + \bm{\alpha}_2 + \bm{\alpha}_4 + k\bm{\alpha}_3.$
  \task $\displaystyle \bm{\alpha}_1 + \bm{\alpha}_2 + \bm{\alpha}_3 + k\bm{\alpha}_4.$
\end{tasks}
\ansat{336;【十年真题】 - 考点:线性方程组的解的结构 - 5}

% --- 题目 ---
\problem[P177-1 (24-1)]{设$\displaystyle \bm{A}$是秩为2的3阶矩阵, $\displaystyle \bm{\alpha}$是满足$\displaystyle \bm{A}\bm{\alpha}=\bm{0}$的非零向量. 若对满足$\displaystyle \bm{\beta}^{\mathrm{T}}\bm{\alpha}=0$的3维列向量$\displaystyle \bm{\beta}$,均有$\displaystyle \bm{A}\bm{\beta}=\bm{\beta}$,则( \quad )
\begin{tasks}(2)
  \task $\displaystyle \bm{A}^3$的迹为2.
  \task $\displaystyle \bm{A}^3$的迹为5.
  \task $\displaystyle \bm{A}^2$的迹为8.
  \task $\displaystyle \bm{A}^2$的迹为9.
\end{tasks}}
\ansat{340;【十年真题】 - 考点:矩阵的特征值和特征向量 - 1}

% --- 题目 ---
\problem[P159-1 (18-1,2,3)]{已知 $\displaystyle a$ 是常数, 且矩阵 $\displaystyle \bm{A} = \begin{pmatrix} 1 & 2 & a \\ 1 & 3 & 0 \\ 2 & 7 & -a \end{pmatrix}$ 可经初等列变换化为矩阵 $\displaystyle \bm{B} = \begin{pmatrix} 1 & a & 2 \\ 0 & 1 & 1 \\ -1 & 1 & 1 \end{pmatrix}.$}
\begin{enumerate}
\item[(1)] 求 $\displaystyle a$;
\item[(2)] 求满足 $\displaystyle \bm{AP} = \bm{B}$ 的可逆矩阵 $\displaystyle \bm{P}$.
\end{enumerate}
\ansat{328;【十年真题】 - 考点二:矩阵方程组的解的情况及求解 - 1}

% --- 题目 ---
\problem[P190-1 (25-3)]{设矩阵$\displaystyle \bm{A}=\begin{pmatrix} 1 & 2 \\ -2 & -a \end{pmatrix}$, $\displaystyle \bm{B}=\begin{pmatrix} 1 & 0 \\ 1 & a \end{pmatrix}$. 若$\displaystyle f(x,y)=|x\bm{A}+y\bm{B}|$是正定二次型,则$\displaystyle a$的取值范围是( \quad )
\begin{tasks}(2)
  \task $(0,2-\sqrt{3})$.
  \task $(2-\sqrt{3},2+\sqrt{3})$.
  \task $(2+\sqrt{3},4)$.
  \task $(0,4)$.
\end{tasks}}
\ansat{350;【十年真题】 - 考点三:正定二次型与正定矩阵 - 1}

% --- 题目 ---
\problem[P185-6 (08-2,3)]{设$\displaystyle \bm{A}$为3阶矩阵, $\displaystyle \bm{\alpha}_1,\bm{\alpha}_2$为$\displaystyle \bm{A}$的分别属于特征值$\displaystyle -1,1$的特征向量, 向量$\displaystyle \bm{\alpha}_3$满足$\displaystyle \bm{A}\bm{\alpha}_3=\bm{\alpha}_2+\bm{\alpha}_3$.
\begin{enumerate}
\item[(1)] 证明$\displaystyle \bm{\alpha}_1,\bm{\alpha}_2,\bm{\alpha}_3$线性无关;
\item[(2)] 令$\displaystyle \bm{P} = \left(\bm{\alpha}_1,\bm{\alpha}_2,\bm{\alpha}_3\right)$,求$\displaystyle \bm{P}^{-1}\bm{A}\bm{P}$.
\end{enumerate}}
\ansat{345;【真题精选】 - 考点:矩阵的相似和相似对角化 - 6}
\vspace{8em}

% --- 题目 ---
\problem[P161-3 (98-3)]{齐次线性方程组}
\[ \begin{cases} \lambda x_1 + x_2 + \lambda^2 x_3 = 0, \\ x_1 + \lambda x_2 + x_3 = 0, \\ x_1 + x_2 + \lambda x_3 = 0 \end{cases} \]
{的系数矩阵记为 $\displaystyle \bm{A}$. 若存在 3 阶矩阵 $\displaystyle \bm{B} \neq \bm{O}$ 使得 $\displaystyle \bm{AB} = \bm{O}$, 则~(~\quad~)}
\begin{tasks}(2)
  \task $\displaystyle \lambda = -2$ 且 $\displaystyle |\bm{B}| = 0$.
  \task $\displaystyle \lambda = -2$ 且 $\displaystyle |\bm{B}| \neq 0$.
  \task $\displaystyle \lambda = 1$ 且 $\displaystyle |\bm{B}| = 0$.
  \task $\displaystyle \lambda = 1$ 且 $\displaystyle |\bm{B}| \neq 0$.
\end{tasks}
\ansat{330;【真题精选】 - 考点一:线性方程组的解的情况及求解 - 3}

% --- 题目 ---
\problem[P164-4 (22-1,2,3)]{设}
\[ \bm{\alpha}_1 = \begin{pmatrix} \lambda \\ 1 \\ 1 \end{pmatrix}, \ \bm{\alpha}_2 = \begin{pmatrix} 1 \\ \lambda \\ 1 \end{pmatrix}, \ \bm{\alpha}_3 = \begin{pmatrix} 1 \\ 1 \\ \lambda \end{pmatrix}, \ \bm{\alpha}_4 = \begin{pmatrix} 1 \\ \lambda \\ \lambda^2 \end{pmatrix}. \]
{若向量组 $\displaystyle \bm{\alpha}_1, \bm{\alpha}_2, \bm{\alpha}_3$ 与 $\displaystyle \bm{\alpha}_1, \bm{\alpha}_2, \bm{\alpha}_4$ 等价, 则 $\displaystyle \lambda$ 的取值范围是~(~\quad~)}
\begin{tasks}(2)
  \task $\displaystyle \{0, 1\}.$
  \task $\displaystyle \{\lambda | \lambda \in \mathbb{R}, \lambda \neq -2\}.$
  \task $\displaystyle \{\lambda | \lambda \in \mathbb{R}, \lambda \neq -1, \lambda \neq -2\}.$
  \task $\displaystyle \{\lambda | \lambda \in \mathbb{R}, \lambda \neq -1\}.$
\end{tasks}
\ansat{333;【十年真题】 - 考点:向量组的线性相关性、线性表示及秩 - 4}

% --- 题目 ---
\problem[P173-1 (11-1,2)]{设 $\displaystyle \bm{A}=(\bm{\alpha}_1, \bm{\alpha}_2, \bm{\alpha}_3, \bm{\alpha}_4)$ 是 4 阶矩阵, $\displaystyle \bm{A}^*$ 为 $\displaystyle \bm{A}$ 的伴随矩阵. 若 $\displaystyle (1,0,1,0)^{\mathrm{T}}$ 是方程组 $\displaystyle \bm{Ax}=\bm{0}$ 的一个基础解系, 则 $\displaystyle \bm{A}^* \bm{x}=\bm{0}$ 的基础解系可为~(~\quad~)}
\begin{tasks}(2)
  \task $\displaystyle \bm{\alpha}_1, \bm{\alpha}_3.$
  \task $\displaystyle \bm{\alpha}_1, \bm{\alpha}_2.$
  \task $\displaystyle \bm{\alpha}_1, \bm{\alpha}_2, \bm{\alpha}_3.$
  \task $\displaystyle \bm{\alpha}_2, \bm{\alpha}_3, \bm{\alpha}_4.$
\end{tasks}
\ansat{338;【真题精选】 - 考点:线性方程组的解的结构 - 1}

% --- 题目 ---
\problem[P156-10 (95-1)]{设 $\displaystyle \bm{A}$ 是 $\displaystyle n$ 阶矩阵, 满足 $\displaystyle \bm{A}\bm{A}^{\mathrm{T}}=\bm{E}$ ($\displaystyle \bm{E}$ 是 $\displaystyle n$ 阶单位矩阵, $\displaystyle \bm{A}^{\mathrm{T}}$ 是 $\displaystyle \bm{A}$ 的转置矩阵), $\displaystyle |\bm{A}| < 0$, 则 $\displaystyle |\bm{A} + \bm{E}| = \underline{\hspace{4em}}. $}
\ansat{327;【真题精选】 - 考点一:矩阵的运算 - 10}

% --- 题目 ---
\problem[P146-4 (20-1,2,3)]{行列式 $\displaystyle \begin{vmatrix} a & 0 & -1 & 1 \\ 0 & a & 1 & -1 \\ -1 & 1 & a & 0 \\ 1 & -1 & 0 & a \end{vmatrix} = \underline{\hspace{4em}}.$}
\ansat{323;【十年真题】 - 考点:具体行列式的计算 - 4}

% --- 题目 ---
\problem[P159-2 (16-1)]{设矩阵
\[ \bm{A} = \begin{pmatrix} 1 & -1 & -1 \\ 2 & a & 1 \\ -1 & 1 & a \end{pmatrix}, \bm{B} = \begin{pmatrix} 2 & 2 \\ 1 & a \\ -a-1 & -2 \end{pmatrix}, \]
当 $\displaystyle a$ 为何值时, 方程 $\displaystyle \bm{AX} = \bm{B}$ 无解、有唯一解、有无穷多解?
在有解时, 求解此方程.}
\ansat{328;【十年真题】 - 考点二:矩阵方程组的解的情况及求解 - 2}

% --- 题目 ---
\problem[P194-1 (05-3)]{设$\displaystyle \bm{D}=\begin{pmatrix} \bm{A} & \bm{C} \\ \bm{C}^{\mathrm{T}} & \bm{B} \end{pmatrix}$为正定矩阵, 其中$\displaystyle \bm{A}, \bm{B}$为$\displaystyle m$阶, $n$阶对称矩阵, $\displaystyle \bm{C}$为$\displaystyle m \times n$矩阵.
\begin{enumerate}
\item[(1)] 计算$\displaystyle \bm{P}^{\mathrm{T}}\bm{D}\bm{P}$, 其中$\displaystyle \bm{P}=\begin{pmatrix} \bm{E}_m & -\bm{A}^{-1}\bm{C} \\ \bm{O} & \bm{E}_n \end{pmatrix}$;
\item[(2)] 利用(1)的结果判断矩阵$\displaystyle \bm{B}-\bm{C}^{\mathrm{T}}\bm{A}^{-1}\bm{C}$是否为正定矩阵, 并证明你的结论.
\end{enumerate}}
\begin{note}
  主要错的是 (2)
\end{note}
\ansat{352;【真题精选】 - 考点三:正定二次型与正定矩阵 - 1}

% --- 题目 ---
\problem[P189-12 (20-1,3)]{设二次型
$$f(x_1,x_2)=x_1^2-4x_1x_2+4x_2^2$$
经正交变换$\displaystyle \begin{pmatrix} x_1 \\ x_2 \end{pmatrix}=\bm{Q}\begin{pmatrix} y_1 \\ y_2 \end{pmatrix}$化为二次型
$$g(y_1,y_2)=ay_1^2+4y_1y_2+by_2^2,$$
其中$\displaystyle a \geqslant b$.
\begin{enumerate}
\item[(1)] 求$\displaystyle a, b$的值;
\item[(2)] 求正交矩阵$\displaystyle \bm{Q}$.
\end{enumerate}}
\ansat{348;【十年真题】 - 考点一:化二次型为标准型 - 12}

% --- 题目 ---
\problem[P160-变式 (04-1)]{设有齐次线性方程组
\[ \begin{cases}
(1+a)x_1 + x_2 + \cdots + x_n = 0, \\
2x_1 + (2+a)x_2 + \cdots + 2x_n = 0, \\
\vdots \\
nx_1 + nx_2 + \cdots + (n+a)x_n = 0
\end{cases} (n \ge 2), \]
试问 $\displaystyle a$ 为何值时, 该方程组有非零解, 并求出其通解.}
\ansat{329;【方法探究】 - 考点一:线性方程组的解的情况及求解 - 变式}

% --- 题目 ---
\problem[P173-9 (93-1)]{设$\displaystyle n$阶矩阵$\displaystyle \bm{A}$的各行元素之和均为零,且$\bm{A}$的秩为$n-1$,则线性方程组$\displaystyle \bm{A}\bm{x}=\bm{0}$的通解为\underline{\hspace{4em}}.}
\ansat{338;【真题精选】 - 考点:线性方程组的解的结构 - 9}

% --- 题目 ---
\problem[P185-9 (04-1,2)]{设矩阵$\displaystyle \bm{A}=\begin{pmatrix} 1 & 2 & -3 \\ -1 & 4 & -3 \\ 1 & a & 5 \end{pmatrix}$的特征方程有一个二重根, 求$\displaystyle a$的值, 并讨论$\displaystyle \bm{A}$是否可相似对角化.}
\ansat{345;【真题精选】 - 考点:矩阵的相似和相似对角化 - 9}

% --- 题目 ---
\problem[P186-10 (02-1)]{设$\displaystyle \bm{A}, \bm{B}$为同阶方阵.
\begin{enumerate}
\item[(1)] 如果$\displaystyle \bm{A}, \bm{B}$相似, 试证$\displaystyle \bm{A}, \bm{B}$的特征多项式相等;
\end{enumerate}}
\ansat{345;【真题精选】 - 考点:矩阵的相似和相似对角化 - 10}

% --- 题目 ---
\problem[P193-3 (14-1,2,3)]{设二次型
$$f(x_1,x_2,x_3)=x_1^2-x_2^2+2ax_1x_3+4x_2x_3$$
的负惯性指数是1, 则$\displaystyle a$的取值范围是\underline{\hspace{4em}}.}
\ansat{351;【真题精选】 - 考点一:化二次型为标准型 - 3}

% --- 题目 ---
\problem[P177-3 (24-3)]{设$\displaystyle \bm{A}$为3阶矩阵, $\displaystyle bm{A}^*$为$\displaystyle \bm{A}$的伴随矩阵, $\displaystyle \bm{E}$为3阶单位矩阵. 若$\displaystyle r(2\bm{E}-\bm{A})=1$, $\displaystyle r(\bm{E}+\bm{A})=2$,则$\displaystyle |\bm{A}^*|=$\underline{\hspace{4em}}.}
\ansat{340;【十年真题】 - 考点:矩阵的特征值和特征向量 - 3}

% --- 题目 ---
\problem[P170-2 (25-2)]{设 3 阶矩阵 $\displaystyle \bm{A}, \bm{B}$ 满足 $\displaystyle r(\bm{AB}) = r(\bm{BA}) + 1$, 则~(~\quad~)}
\begin{tasks}(1)
  \task 方程组 $\displaystyle (\bm{A} + \bm{B})\bm{x} = \bm{0}$ 只有零解.
  \task 方程组 $\displaystyle \bm{A}\bm{x} = \bm{0}$ 与方程组 $\displaystyle \bm{B}\bm{x} = \bm{0}$ 均只有零解.
  \task 方程组 $\displaystyle \bm{A}\bm{x} = \bm{0}$ 与方程组 $\displaystyle \bm{B}\bm{x} = \bm{0}$ 没有公共非零解.
  \task 方程组 $\displaystyle \bm{ABA}\bm{x} = \bm{0}$ 与方程组 $\displaystyle \bm{BAB}\bm{x} = \bm{0}$ 有公共非零解.
\end{tasks}
\ansat{336;【十年真题】 - 考点:线性方程组的解的结构 - 2}

% --- 题目 ---
\problem[P180-4 (22-1)]{下述四个条件中, 3阶矩阵$\displaystyle \bm{A}$可对角化的一个充分但不必要条件是( \quad )
\begin{tasks}(2)
  \task $\displaystyle \bm{A}$有3个互不相等的特征值.
  \task $\displaystyle \bm{A}$有3个线性无关的特征向量.
  \task $\displaystyle \bm{A}$有3个两两线性无关的特征向量.
  \task $\displaystyle \bm{A}$的属于不同特征值的特征向量正交.
\end{tasks}}
\ansat{341;【十年真题】 - 考点:矩阵的相似和相似对角化 - 4}

% --- 题目 ---
\problem[P164-5 (21-1)]{已知 $\displaystyle \bm{\alpha}_1 = \begin{pmatrix} 1 \\ 0 \\ 1 \end{pmatrix}, \bm{\alpha}_2 = \begin{pmatrix} 1 \\ 2 \\ 1 \end{pmatrix}, \bm{\alpha}_3 = \begin{pmatrix} 3 \\ 1 \\ 2 \end{pmatrix}$, 记 $\displaystyle \bm{\beta}_1 = \bm{\alpha}_1, \ \bm{\beta}_2 = \bm{\alpha}_2 - k\bm{\beta}_1, \ \bm{\beta}_3 = \bm{\alpha}_3 - l_1\bm{\beta}_1 - l_2\bm{\beta}_2$. 若 $\displaystyle \bm{\beta}_1, \ \bm{\beta}_2, \ \bm{\beta}_3$ 两两正交, 则 $\displaystyle l_1, l_2$ 依次为~(~\quad~)}
\begin{tasks}(2)
  \task $\displaystyle \frac{5}{2}, \frac{1}{2}.$
  \task $\displaystyle -\frac{5}{2}, \frac{1}{2}.$
  \task $\displaystyle \frac{5}{2}, -\frac{1}{2}.$
  \task $\displaystyle -\frac{5}{2}, -\frac{1}{2}.$
\end{tasks}
\ansat{333;【十年真题】 - 考点:向量组的线性相关性、线性表示及秩 - 5}
\vspace{5em}

% --- 题目 ---
\problem[P150-2 (24-2)]{设 $\displaystyle \bm{A}$ 为 4 阶矩阵, $\displaystyle \bm{A}^*$ 为 $\displaystyle \bm{A}$ 的伴随矩阵. 若 $\displaystyle \bm{A}(\bm{A} - \bm{A}^*) = \bm{O}$, 且 $\displaystyle \bm{A} \neq \bm{A}^*$, 则 $\displaystyle r(\bm{A})$ 取值为~(~\quad~)}
\begin{tasks}(2)
  \task $\displaystyle 0$ 或 $\displaystyle 1$.
  \task $\displaystyle 1$ 或 $\displaystyle 3$.
  \task $\displaystyle 2$ 或 $\displaystyle 3$.
  \task $\displaystyle 1$ 或 $\displaystyle 2$.
\end{tasks}
\ansat{325;【十年真题】 - 考点三:矩阵的秩与等价 - 2}

% --- 题目 ---
\problem[P181-17 (16-1,2,3)]{已知矩阵$\displaystyle \bm{A}=\begin{pmatrix} 0 & -1 & 1 \\ 2 & -3 & 0 \\ 0 & 0 & 0 \end{pmatrix}$.
\begin{enumerate}
\item[(1)] 求$\displaystyle \bm{A}^{99}$;
\item[(2)] 设3阶矩阵$\displaystyle \bm{B}=(\bm{\alpha}_1,\bm{\alpha}_2,\bm{\alpha}_3)$满足$\displaystyle \bm{B}^2=\bm{BA}$. 记$\displaystyle \bm{B}^{100}=\left(\bm{\beta}_1,\bm{\beta}_2,\bm{\beta}_3\right)$, 将$\displaystyle \bm{\beta}_1,\bm{\beta}_2,\bm{\beta}_3$分别表示为$\displaystyle \bm{\alpha}_1,\bm{\alpha}_2,\bm{\alpha}_3$的线性组合.
\end{enumerate}}
\ansat{343;【十年真题】 - 考点:矩阵的相似和相似对角化 - 17}

% --- 题目 ---
\problem[P170-3 (22-1)]{设 $\displaystyle \bm{A}, \bm{B}$ 为 $\displaystyle n$ 阶矩阵, $\displaystyle \bm{E}$ 为单位矩阵. 若方程组 $\displaystyle \bm{Ax} = \bm{0}$ 与 $\displaystyle \bm{Bx} = \bm{0}$ 同解, 则~(~\quad~)}
\begin{tasks}(1)
  \task 方程组 $\displaystyle \begin{pmatrix} \bm{A} & \bm{O} \\ \bm{E} & \bm{B} \end{pmatrix} \bm{y} = \bm{0}$ 只有零解.
  \task 方程组 $\displaystyle \begin{pmatrix} \bm{E} & \bm{A} \\ \bm{O} & \bm{AB} \end{pmatrix} \bm{y} = \bm{0}$ 只有零解.
  \task 方程组 $\displaystyle \begin{pmatrix} \bm{A} & \bm{B} \\ \bm{O} & \bm{B} \end{pmatrix} \bm{y} = \bm{0}$ 与 $\displaystyle \begin{pmatrix} \bm{B} & \bm{A} \\ \bm{O} & \bm{A} \end{pmatrix} \bm{y} = \bm{0}$ 同解.
  \task 方程组 $\displaystyle \begin{pmatrix} \bm{AB} & \bm{B} \\ \bm{O} & \bm{A} \end{pmatrix} \bm{y} = \bm{0}$ 与 $\displaystyle \begin{pmatrix} \bm{BA} & \bm{A} \\ \bm{O} & \bm{B} \end{pmatrix} \bm{y} = \bm{0}$ 同解.
\end{tasks}
\ansat{336;【十年真题】 - 考点:线性方程组的解的结构 - 3}
\vspace{10em}

% --- 题目 ---
\problem[P179-36 (96-1)]{设$\displaystyle \bm{A}=\bm{E}-\bm{\xi}\bm{\xi}^{\mathrm{T}}$, 其中$\displaystyle \bm{E}$是$\displaystyle n$阶单位矩阵, $\displaystyle \bm{\xi}$是$n$维非零列向量, $\displaystyle \bm{\xi}^{\mathrm{T}}$是$\displaystyle \bm{\xi}$的转置. 证明:
\begin{enumerate}
\item[(1)] $\displaystyle \bm{A}^2=\bm{A}$的充分条件是$\displaystyle \bm{\xi}^{\mathrm{T}}\bm{\xi}=1$
\item[(2)] 当$\displaystyle \bm{\xi}^{\mathrm{T}}\bm{\xi}=1$时, $\displaystyle \bm{A}$是不可逆矩阵.
\end{enumerate}}
\ansat{341;【真题精选】 - 考点:矩阵的特征值和特征向量 - 6}

% --- 题目 ---
\problem[P184-变式 2 (11-1,2,3)]{设$\displaystyle \bm{A}$为3阶实对称矩阵, $\displaystyle \bm{A}$的秩为2,且
$$\bm{A}\begin{pmatrix} 1 & 1 \\ 0 & 0 \\ -1 & 1 \end{pmatrix} = \begin{pmatrix} -1 & 1 \\ 0 & 0 \\ 1 & 1 \end{pmatrix}.$$
\begin{enumerate}
\item[(1)] 求$\displaystyle \bm{A}$的所有特征值与特征向量;
\item[(2)] 求矩阵$\displaystyle \bm{A}$.
\end{enumerate}}
\ansat{344;【方法探究】 - 考点:矩阵的相似和相似对角化 - 变式2}

% --- 题目 ---
\problem[P180-2 (24-2)]{设$\displaystyle \bm{A},\bm{B}$为2阶矩阵, 且$\displaystyle \bm{AB}=\bm{BA}$, 则“$\displaystyle \bm{A}$有两个不相等的特征值”是“$\displaystyle \bm{B}$可对角化”的( \quad )
\begin{tasks}(2)
  \task 充分必要条件.
  \task 充分不必要条件.
  \task 必要不充分条件.
  \task 既不充分也不必要条件.
\end{tasks}}
\ansat{341;【十年真题】 - 考点:矩阵的相似和相似对角化 - 2}

% --- 题目 ---
\problem[P177-5 (18-1)]{设2阶矩阵$\displaystyle \bm{A}$有两个不同特征值, $\displaystyle \bm{\alpha}_1,\bm{\alpha}_2$ 是 $\displaystyle \bm{A}$的线性无关的特征向量, 且满足
$$\bm{A}^2(\bm{\alpha}_1 + \bm{\alpha}_2) = \bm{\alpha}_1 + \bm{\alpha}_2,$$
则$\displaystyle |\bm{A}|=$\underline{\hspace{4em}}.}
\ansat{340;【十年真题】 - 考点:矩阵的特征值和特征向量 - 5}
\vspace{5em}

% --- 题目 ---
\problem[P150-4 (22-2,3)]{设 $\displaystyle \bm{A}$ 为 3 阶矩阵, 交换 $\displaystyle \bm{A}$ 的第 2 行和第 3 行, 再将第 2 列的 $\displaystyle -1$ 倍加到第 1 列, 得到 $\displaystyle \begin{pmatrix} -2 & 1 & -1 \\ 1 & -1 & 0 \\ -1 & 0 & 0 \end{pmatrix}$, 则 $\displaystyle \bm{A}^{-1}$ 的迹 $\displaystyle \mathrm{tr}(\bm{A}^{-1}) = \underline{\hspace{4em}}.$}
\ansat{325;【十年真题】 - 考点二:矩阵的初等变换与初等矩阵 - 4}

% --- 题目 ---
\problem[P163-2 (23-1)]{已知 $\displaystyle n$ 阶矩阵 $\displaystyle \bm{A}, \bm{B}, \bm{C}$ 满足 $\displaystyle \bm{ABC} = \bm{O}, \bm{E}$ 为 $\displaystyle n$ 阶单位矩阵. 记矩阵
$$ 
\begin{pmatrix} \bm{O} & \bm{A} \\ \bm{BC} & \bm{E} \end{pmatrix}, \; \begin{pmatrix} \bm{AB} & \bm{C} \\ \bm{O} & \bm{E} \end{pmatrix}, \; \begin{pmatrix} \bm{E} & \bm{AB} \\ \bm{AB} & \bm{O} \end{pmatrix}
$$ 
的秩分别为 $\displaystyle r_1, r_2, r_3$, 则~(~\quad~)}
\begin{tasks}(2)
  \task $\displaystyle r_1 \le r_2 \le r_3.$
  \task $\displaystyle r_1 \le r_3 \le r_2.$
  \task $\displaystyle r_3 \le r_1 \le r_2.$
  \task $\displaystyle r_2 \le r_1 \le r_3.$
\end{tasks}
\ansat{333;【十年真题】 - 考点:向量组的线性相关性、线性表示及秩 - 2}

% --- 题目 ---
\problem[P181-11 (18-2)]{设$\displaystyle \bm{A}$为3阶矩阵, $\displaystyle \bm{\alpha}_1,\bm{\alpha}_2,\bm{\alpha}_3$为线性无关的向量组. 若$\displaystyle \bm{A}\bm{\alpha}_1=2\bm{\alpha}_1+\bm{\alpha}_2+\bm{\alpha}_3$, $\displaystyle \bm{A}\bm{\alpha}_2=\bm{\alpha}_2+2\bm{\alpha}_3$, $\displaystyle \bm{A}\bm{\alpha}_3=-\bm{\alpha}_2+\bm{\alpha}_3$,则$\bm{A}$的实特征值为\underline{\hspace{4em}}.}
\ansat{342;【十年真题】 - 考点:矩阵的相似和相似对角化 - 11}

% --- 题目 ---
\problem[P173-7 (04-4)]{设 $\displaystyle \bm{A} = \left(a_{ij}\right)_{3 \times 3}$ 是实正交矩阵, 且 $\displaystyle a_{11} = 1, \bm{b} = (1, 0, 0)^{\mathrm{T}}$, 则线性方程组 $\displaystyle \bm{Ax} = \bm{b}$ 的解是 \underline{\hspace{4em}}.}
\ansat{338;【真题精选】 - 考点:线性方程组的解的结构 - 7}

% \nocite{*}

% \printbibliography[heading=bibintoc, title=\ebibname]
% \appendix
% \chapter{答案}

\end{document}
