% --- 题目 ---
\problem[李武六-1-1]{当 $\displaystyle x \to 0$ 时, 无穷小量
  $\displaystyle a_1 = \int_x^{2\sin x} (e^{t^2} - 1) \, \mathrm{d}t$, 
  $\displaystyle a_2 = \int_x^{e^x - 1} \ln \cos t \, \mathrm{d}t$, 
  $\displaystyle a_3 = \int_{x^2}^x \frac{\tan^3 t}{t} \, \mathrm{d}t$ 
  关于 $\displaystyle x$ 的阶数分别为~(~\quad~)
\begin{tasks}(4)
  \task 2, 3, 4.
  \task 3, 3, 3.
  \task 3, 5, 3.
  \task 3, 4, 3.
\end{tasks}}

% --- 题目 ---
\problem[李武六-1-3]{在 $\displaystyle Oxy$ 平面上, 光滑曲线 $\displaystyle L$ 过 $\displaystyle (1,0)$ 点, 并且曲线上任意一点 $\displaystyle P(x,y) (x \ne 0)$ 处的切线斜率与直线 $\displaystyle OP$ 的斜率之差等于 $\displaystyle ax $($a>0$ 为常数). 如果 $\displaystyle L$ 与直线 $\displaystyle y=ax$ 所围成的平面图形的面积为 8, 则 $\displaystyle a$ 的值为~(~\quad~)
\begin{tasks}(4)
  \task 2.
  \task 4.
  \task 6.
  \task 8.
\end{tasks}}

\problem[李武六-1-8]{设连续型随机变量 $\displaystyle X$ 的分布函数为 $\displaystyle F(x)$, 且 $\displaystyle F(0)=0$. 则下列函数可作为分布函数的是~(~\quad~)
\begin{tasks}(2)
  \task $\displaystyle G_1(x) = \begin{cases} 1+F\left(\frac{1}{x}\right), & x > 1, \\ 0, & x \le 1. \end{cases}$
  \task $\displaystyle G_2(x) = \begin{cases} 1-F\left(\frac{1}{x}\right), & x > 1, \\ 0, & x \le 1. \end{cases}$
  \task $\displaystyle G_3(x) = \begin{cases} F(x)-F\left(\frac{1}{x}\right), & x > 1, \\ 0, & x \le 1. \end{cases}$
  \task $\displaystyle G_4(x) = \begin{cases} F(x)+F\left(\frac{1}{x}\right), & x > 1, \\ 0, & x \le 1. \end{cases}$
\end{tasks}}

% --- 题目 ---
\problem[李武六-1-10]{一颗陨石等可能地坠落在区域 $\displaystyle A_1, A_2, A_3, A_4$ 后, 有关部门千方百计地要找到它. 根据现有的搜索条件, 如果陨石坠落在 $\displaystyle A_i$, 则在该区域被找到的概率是 $\displaystyle p_i$ (这里 $\displaystyle p_i$ 是由 $\displaystyle A_i$ 的地貌条件决定的; $i=1,2,3,4$). 现对 $A_1$ 搜索后没有发现这块陨石, 则陨石坠落在 $\displaystyle A_4$ 的概率为~(~\quad~)
\begin{tasks}(4)
  \task $\displaystyle \frac{1}{3}$.
  \task $\displaystyle \frac{1}{4}$.
  \task $\displaystyle \frac{1-p_1}{4-p_1}$.
  \task $\displaystyle \frac{1}{4-p_1}$.
\end{tasks}}

% --- 题目 ---
\problem[李武六-1-11]{设 $\displaystyle f(x)$ 是定义在 $\displaystyle (-\infty, +\infty)$ 上以 $\displaystyle 2\pi$ 为周期的二阶可导函数, 且满足等式 $\displaystyle f(x) + 2f'(x+\pi) = \sin x$,则 $\displaystyle f(x) = \underline{\hspace{6em}}.$}

% --- 题目 ---
\problem[李武六-1-12]{设 $\displaystyle f(x)$ 在 $\displaystyle [-1,1]$ 上连续, 且满足
\[ f(x) = x^2 + \mathrm{e}^{-3x^2} \ln(x + \sqrt{1+x^2}) + [1 - \sin^6(\pi x)] \int_{-1}^1 f(x) \mathrm{d}x, \]
则 $\displaystyle \int_{-1}^1 f(x) \mathrm{d}x = \underline{\hspace{4em}}.$}

% --- 题目 ---
\problem[李武六-1-13]{设当 $\displaystyle x > 0$ 时, 方程 $\displaystyle kx + \frac{675}{x^2} = 2025$ 有且仅有一个根, 则 $\displaystyle k$ 的取值范围是 $\underline{\hspace{4em}}.$}

% --- 题目 ---
\problem[李武六-1-14]{设 $\displaystyle a > 0, b > 0, f(x, y) = \max\{\mathrm{e}^{b^2x^2}, \mathrm{e}^{a^2y^2}\}$, 则 $\displaystyle \int_0^a \mathrm{d}x \int_0^b f(x, y) \mathrm{d}y = \underline{\hspace{4em}}.$}

% --- 题目 ---
\problem[李武六-1-15]{设 $\displaystyle \bm{A}$ 为三阶矩阵, $\displaystyle \boldsymbol{\alpha}, \boldsymbol{\beta}$ 为三维列向量. 已知 $\displaystyle \boldsymbol{\alpha}, \boldsymbol{\beta}$ 线性无关, 且 $\displaystyle \bm{A}\boldsymbol{\alpha} = 2\boldsymbol{\beta}$, $\displaystyle \bm{A}\boldsymbol{\beta} = 2\boldsymbol{\alpha}$. 记 $\displaystyle f(\lambda) = |\lambda\bm{E} - \bm{A}|$. 若 $\displaystyle f(0) = 12$, 则 $\displaystyle f(5) = \underline{\hspace{4em}}.$}

% --- 题目 ---
\problem[李武六-1-17]{求极限
\[ \lim_{x \to 0} \frac{2\ln(2 - \cos x) - 3\left[\left(1 + \sin^2 x\right)^{\frac{1}{3}} - 1\right]}{x^2\left[\ln(1 + x) + \ln(1 - x)\right]}. \]}

% --- 题目 ---
\problem[李武六-1-19]{计算
\[ I = \iint\limits_D \left( x^3 \cos y + x^2 + y^2 - \sin x - 2y + 1 \right) \mathrm{d}\sigma, \]
其中
\[ D = \left\{ (x,y) | 1 \le x^2 + (y-1)^2 \le 2, x^2 + y^2 \le 1 \right\}. \]}

% --- 题目 ---
\probtagged[李武六-1-20]{已知
\[ \lim_{n \to \infty} \frac{\int_0^{n\pi} x |\sin x| \mathrm{d}x}{n^\alpha} = A \neq 0. \]
\begin{enumerate}
\item[(1)] 试确定 $\displaystyle \alpha$ 和 $\displaystyle A$ 的值.
\item[(2)] 证明级数
\[ \sum_{n=1}^\infty \frac{(-1)^n n^{\alpha-1}}{\int_0^{n\pi} x |\sin x| \mathrm{d}x} \]
收敛, 并求其和.
\end{enumerate}}

% --- 题目 ---
\problem[李武六-1-21]{已知二次型
\[ f(x_1, x_2, x_3) = (x_1 + x_3)^2 + (x_1 + 2x_2 + ax_3)^2 + (x_1 - ax_2 - 2x_3)^2. \]
\begin{enumerate}
\item[(1)] 求方程 $\displaystyle f(x_1, x_2, x_3) = 0$ 的解.
\item[(2)] 求 $\displaystyle f(x_1, x_2, x_3)$ 的规范形.
\item[(3)] 当 $\displaystyle f(x_1, x_2, x_3) = 0$ 有非零解时, 确定常数 $\displaystyle a$, 使矩阵
\[ \bm{A} = \begin{pmatrix} 3 & 1 & 2 \\ 1 & a & -2 \\ 2 & -2 & 9 \end{pmatrix} \]
为正定矩阵, 并求二次型 $\displaystyle g(\bm{x}) = \bm{x}^{\mathrm{T}}\bm{A}\bm{x}$ 在 $\displaystyle \bm{x}^{\mathrm{T}}\bm{x} = 2$ 下的最大值.
\end{enumerate}}

% --- 题目 ---
\problem[李武六-1-22]{设总体 $\displaystyle X \sim N(\alpha + \beta, \sigma^2)$, $\displaystyle Y \sim N(\alpha - \beta, \sigma^2)$, $\displaystyle X$ 和 $\displaystyle Y$ 相互独立.
\begin{enumerate}
\item[(1)] 若 $\displaystyle \alpha, \beta$ 未知, $\displaystyle \sigma^2$ 已知. $\displaystyle X_1, X_2, \dots, X_n$ 和 $\displaystyle Y_1, Y_2, \dots, Y_n$ 分别是总体 $\displaystyle X$ 和 $\displaystyle Y$ 的简单随机样本, 试求 $\displaystyle \alpha, \beta$ 的矩估计量和最大似然估计量.
\item[(2)] 求 (1) 中矩估计量及最大似然估计量的数学期望和方差.
\item[(3)] 当 $\displaystyle \alpha, \beta, \sigma^2$ 为何值时, 可使 $\displaystyle \left({X} + {Y}\right)^2$ 服从 $\displaystyle \chi^2$ 分布?
\end{enumerate}}

% --- 题目 ---
\problem[李武六-2-1]{记符号函数 $\displaystyle \mathrm{sgn} \ x = \begin{cases} 1, & x > 0, \\ 0, & x = 0, \\ -1, & x < 0, \end{cases}$
则函数 $\displaystyle f(x) = \mathrm{sgn}\left(\sin \frac{\pi}{x}\right)$ 的间断点为~(~\quad~)
\begin{tasks}(1)
  \task 一个第一类间断点及一个第二类间断点.
  \task 无穷个第一类间断点及一个第二类间断点.
  \task 一个第一类间断点及无穷个第二类间断点.
  \task 只有一个间断点.
\end{tasks}}

% --- 题目 ---
\probtagged[李武六-2-4]{设 $\displaystyle f(x,y)$ 连续, 且 $\displaystyle f(x,y) = \mathrm{e}^{x^2+y^2} + xy \iint\limits_D f(x,y) \mathrm{d}x \mathrm{d}y$, 其中 $\displaystyle D = \{(x,y) | 0 \le x \le 1, 0 \le y \le 1\}$, 则 $\displaystyle \frac{\partial^2 f}{\partial x \partial y}$ 等于~(~\quad~)
\begin{tasks}(2)
  \task $\displaystyle 4xy\mathrm{e}^{x^2+y^2} + \frac{9}{16}(\mathrm{e}-1)^2.$
  \task $\displaystyle 2xy\mathrm{e}^{x^2+y^2} + \frac{9}{16}(\mathrm{e}-1).$
  \task $\displaystyle 4xy\mathrm{e}^{x^2+y^2} + \frac{9}{32}(\mathrm{e}-1)^2.$
  \task $\displaystyle 4xy\mathrm{e}^{x^2+y^2} + \frac{9}{16}(\mathrm{e}-1).$
\end{tasks}}

% --- 题目 ---
\problem[李武六-2-5]{已知三阶矩阵 $\displaystyle \bm{A}, \bm{B}$ 满足 $\displaystyle \bm{A} - \bm{B} = \bm{AB}$, 则在下面 \textcircled{1} $\displaystyle \bm{A}$ 与 $\displaystyle \bm{B}$ 等价; \textcircled{2} $\displaystyle \bm{A}$ 可逆等价于 $\bm{B}$ 可逆; \textcircled{3} $\displaystyle \displaystyle \bm{BA} = \bm{A} - \bm{B}$ 三个结论中, 正确的结论个数是~(~\quad~)
\begin{tasks}(4)
  \task 0.
  \task 1.
  \task 2.
  \task 3.
\end{tasks}}

% --- 题目 ---
\problem[李武六-2-7]{设 $\displaystyle \boldsymbol{\alpha} = (a_1, a_2, a_3)^{\mathrm{T}}$, $\displaystyle \boldsymbol{\beta} = (b_1, b_2, b_3)^{\mathrm{T}}$. 已知 $\displaystyle \boldsymbol{\alpha}, \boldsymbol{\beta}$ 正交且为单位向量, 则二次型 $\displaystyle f(x_1, x_2, x_3) = (a_1x_1 + a_2x_2 + a_3x_3)(b_1x_1 + b_2x_2 + b_3x_3)$ 的秩为~(~\quad~)
\begin{tasks}(4)
  \task 3.
  \task 2.
  \task 1.
  \task 0.
\end{tasks}}

% --- 题目 ---
\problem[李武六-2-8]{一个袋子中装有白球和黑球, 有放回地取 $\displaystyle n$ 次, 其中有 $\displaystyle k$ 个白球, 则袋子中黑球数和白球数之比 $\displaystyle R$ 的最大似然估计为 ( \quad )
\begin{tasks}(4)
  \task $\displaystyle \hat{R} = \frac{n}{k} - 1.$
  \task $\displaystyle \hat{R} = \frac{k}{n}.$
  \task $\displaystyle \hat{R} = \frac{n}{k}.$
  \task $\displaystyle \hat{R} = 1 - \frac{k}{n}.$
\end{tasks}}

% --- 题目 ---
\problem[李武六-2-11]{$\displaystyle \int \arctan(1+\sqrt{x}) \mathrm{d}x = \underline{\hspace{4em}}.$}

% --- 题目 ---
\problem[李武六-2-13]{已知函数 $\displaystyle f(t)$ 满足 $\displaystyle tf(t) = 1 + \int_0^t s^2 f(s) \mathrm{d}s$, 则 $\displaystyle f(x) = \underline{\hspace{4em}}.$}

% --- 题目 ---
\probtagged[李武六-2-15]{设 $\displaystyle \bm{A}, \bm{B}$ 均为 $\displaystyle n$ 阶方阵, 且 $\displaystyle \bm{E} - \bm{AB}$ 可逆, 则 $\displaystyle (\bm{E} - \bm{BA})^{-1} = \underline{\hspace{4em}}.$}