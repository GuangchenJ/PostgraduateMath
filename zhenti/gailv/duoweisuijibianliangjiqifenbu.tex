% --- 题目 ---
\problem[P206-1 (24-1,3)]{设随机变量 $\displaystyle X, Y$ 相互独立, 且均服从参数为 $\displaystyle \lambda$ 的指数分布. 令 $\displaystyle Z = |X - Y|$, 则下列随机变量中与 $\displaystyle Z$ 同分布的是~(~\quad~)}
\begin{tasks}(2)
  \task $\displaystyle X + Y.$
  \task $\displaystyle \frac{X+Y}{2}.$
  \task $\displaystyle 2X.$
  \task $\displaystyle X.$
\end{tasks}
\ansat{358;【十年真题】 - 考点二:两个随机变量的函数的分布 - 1}

% --- 题目 ---
\problem[P206-2 (23-1-局部)]{设二维随机变量 $\displaystyle (X, Y)$ 的概率密度为
\[ f(x, y) = \begin{cases} \displaystyle \frac{2}{\pi}(x^2 + y^2), & \displaystyle x^2 + y^2 \le 1, \\ 0, & \text{其他}. \end{cases} \]
\begin{enumerate}
\item[(1)] $\displaystyle X$ 与 $\displaystyle Y$ 是否相互独立?
\item[(2)] 求 $\displaystyle Z = X^2 + Y^2$ 的概率密度.
\end{enumerate}}
\ansat{358;【十年真题】 - 考点二:两个随机变量的函数的分布 - 2}
\vspace{5em}

% --- 题目 ---
\problem[P206-3 (20-1)]{设随机变量 $\displaystyle X_1, X_2, X_3$ 相互独立, 其中 $\displaystyle X_1$ 与 $\displaystyle X_2$ 均服从标准正态分布, $\displaystyle X_3$ 的概率分布为 $\displaystyle P\{X_3=0\} = P\{X_3=1\} = \frac{1}{2}$. $\displaystyle Y = X_3 X_1 + (1-X_3)X_2$.
\begin{enumerate}
\item[(1)] 求二维随机变量 $\displaystyle (X_1, Y)$ 的分布函数, 结果用标准正态分布函数 $\displaystyle \Phi(x)$ 表示;
\item[(2)] 证明随机变量 $\displaystyle Y$ 服从标准正态分布.
\end{enumerate}}
\begin{note}
  主要错的是 (2)
\end{note}
\ansat{358;【十年真题】 - 考点二:两个随机变量的函数的分布 - 3}

% --- 题目 ---
\problem[5. (16-1,3)]{设二维随机变量 $\displaystyle (X,Y)$ 在区域
\[ D = \{(x,y) \ | \ 0 < x < 1, \ x^2 < y < \sqrt{x}\} \]
上服从均匀分布, 令
\[ U = \begin{cases} 1, & X \le Y, \\ 0, & X > Y. \end{cases} \].
\begin{enumerate}
\item[(1)] 写出 $\displaystyle (X,Y)$ 的概率密度;
\item[(2)] 问 $\displaystyle U$ 与 $\displaystyle X$ 是否相互独立? 并说明理由;
\item[(3)] 求 $\displaystyle Z = U + X$ 的分布函数 $\displaystyle F(z)$.
\end{enumerate}}
\ansat{358;【十年真题】 - 考点二:两个随机变量的函数的分布 - 5}

% --- 题目 ---
\problem[P210-1 (12-1)]{设随机变量 $\displaystyle X$ 与 $\displaystyle Y$ 相互独立, 且分别服从参数为 $\displaystyle 1$ 与参数为 $\displaystyle 4$ 的指数分布, 则 $\displaystyle P\{X < Y\} =~(~\quad~)$
}
\begin{tasks}(2)
  \task $\displaystyle \frac{1}{5}.$
  \task $\displaystyle \frac{1}{3}.$
  \task $\displaystyle \frac{2}{5}.$
  \task $\displaystyle \frac{4}{5}.$
\end{tasks}
\ansat{360;【真题精选】 - 考点一:二维随机变量的分布 - 1}

% --- 题目 ---
\problem[P210-4 (10-1,3)]{设二维随机变量 $\displaystyle (X, Y)$ 的概率密度为 $\displaystyle f(x, y) = A\mathrm{e}^{-2x^2 + 2xy - y^2}, -\infty < x < +\infty, -\infty < y < +\infty$, 求常数 $\displaystyle A$ 及条件概率密度 $\displaystyle f_{Y|X}(y|x)$.}
\ansat{360;【真题精选】 - 考点一:二维随机变量的分布 - 4}

% --- 题目 ---
\problem[P211-5 (09-1,3)]{袋中有 $\displaystyle 1$ 个红球, $\displaystyle 2$ 个黑球与 $\displaystyle 3$ 个白球, 现有放回地从袋中取两次, 每次取一个球. 以 $\displaystyle X, Y, Z$ 分别表示两次取球所取得的红球、黑球与白球的个数.
\begin{enumerate}
\item[(1)] 求 $\displaystyle P\{X=1|Z=0\}$;
\item[(2)] 求二维随机变量 $\displaystyle (X,Y)$ 的概率分布.
\end{enumerate}}
\ansat{360;【真题精选】 - 考点一:二维随机变量的分布 - 5}

% --- 题目 ---
\problem[P211-7 (01-1)]{设某班车起点站上车人数 $\displaystyle X$ 服从参数为 $\displaystyle \lambda \ (\lambda > 0)$ 的泊松分布, 每位乘客在中途下车的概率为 $\displaystyle p \ (0 < p < 1)$, 且中途下车与否相互独立. 以 $\displaystyle Y$ 表示在中途下车的人数, 求:
\begin{enumerate}
\item[(1)] 在发车时有 $\displaystyle n$ 个乘客的条件下, 中途有 $\displaystyle m$ 人下车的概率.
\item[(2)] 二维随机变量 $\displaystyle (X, Y)$ 的概率分布.
\end{enumerate}}
\ansat{360;【真题精选】 - 考点一:二维随机变量的分布 - 5}

% --- 题目 ---
\problem[P211-9 (95-3)]{已知随机变量 $\displaystyle X$ 和 $\displaystyle Y$ 的联合概率密度为
\[ \varphi(x, y) = \begin{cases} 4xy, & 0 \le x \le 1, 0 \le y \le 1, \\ 0, & \text{其他}, \end{cases} \]
求 $\displaystyle X$ 和 $\displaystyle Y$ 联合分布函数 $\displaystyle F(x, y)$.}
\ansat{360;【真题精选】 - 考点一:二维随机变量的分布 - 9}

% --- 题目 ---
\problem[P212-3 (07-1,3)]{设二维随机变量 $\displaystyle (X, Y)$ 的概率密度为
\[ f(x, y) = \begin{cases} 2 - x - y, & 0 < x < 1, 0 < y < 1, \\ 0, & \text{其他}. \end{cases} \]
\begin{enumerate}
\item[(1)] 求 $\displaystyle P\{X > 2Y\}$;
\item[(2)] 求 $\displaystyle Z = X + Y$ 的概率密度 $\displaystyle f_Z(z)$.
\end{enumerate}}
\ansat{361;【真题精选】 - 考点二:两个随机变量的函数的分布 - 3}

% --- 题目 ---
\problem[P212-5 (01-3)]{设随机变量 $\displaystyle X$ 和 $\displaystyle Y$ 的联合分布是正方形 $\displaystyle G = \left\{ (x, y) | 1 \le x \le 3, 1 \le y \le 3 \right\}$ 上的均匀分布, 试求随机变量 $\displaystyle U = |X - Y|$ 的概率密度 $\displaystyle p(u)$.}
\ansat{361;【真题精选】 - 考点二:两个随机变量的函数的分布 - 5}
