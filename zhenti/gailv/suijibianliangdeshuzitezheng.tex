% --- 题目 ---
\problem[P214-1 (24-3)]{设随机变量 $\displaystyle X$ 的概率密度为
\[ f(x) = \begin{cases} 6x(1-x), & 0 < x < 1, \\ 0, & \text{其他}, \end{cases} \]
则 $\displaystyle X$ 的三阶中心矩 $\displaystyle E[(X-EX)^3] = (\quad)$
}
\begin{tasks}(4)
  \task $\displaystyle -\frac{1}{32}.$
  \task $\displaystyle 0.$
  \task $\displaystyle \frac{1}{16}.$
  \task $\displaystyle \frac{1}{2}.$
\end{tasks}
\ansat{362;【十年真题】 - 考点一:随机变量的数学期望与方差 - 1}

% --- 题目 ---
\problem[P214-7 (17-1)]{设随机变量 $\displaystyle X$ 的分布函数为 $\displaystyle F(x) = 0.5\Phi(x) + 0.5\Phi\left(\frac{x-4}{2}\right)$, 其中 $\displaystyle \Phi(x)$ 为标准正态分布函数, 则 $\displaystyle EX = \underline{\hspace{4em}}. $}
\ansat{362;【十年真题】 - 考点一:随机变量的数学期望与方差 - 7}

% --- 题目 ---
\problem[P214-8 (25-1,3)]{投保人的损失事件发生时, 保险公司的赔付额 $\displaystyle Y$ 与投保人的损失额 $\displaystyle X$ 的关系为
\[ Y = \begin{cases} 0, & X \le 100, \\ X - 100, & X > 100. \end{cases} \]
设损失事件发生时, 投保人的损失额 $\displaystyle X$ 的概率密度为
\[ f(x) = \begin{cases} \displaystyle \frac{2 \times 100^2}{(100+x)^3}, & x > 0, \\ 0, & x \le 0. \end{cases} \]
\begin{enumerate}
\item[(1)] 求 $\displaystyle P\{Y > 0\}$ 及 $\displaystyle EY$;
\item[(2)] 这种损失事件在一年内发生的次数记为 $\displaystyle N$, 保险公司在一年内就这种损失事件产生的理赔次数记为 $\displaystyle M$. 假设 $\displaystyle N$ 服从参数为 8 的泊松分布, 在 $\displaystyle N=n \ (n \ge 1)$ 的条件下, $\displaystyle M$ 服从二项分布 $\displaystyle B(n, p)$, 其中 $\displaystyle p=P\{Y > 0\}$. 求 $\displaystyle M$ 的概率分布.
\end{enumerate}}
\ansat{363;【十年真题】 - 考点一:随机变量的数学期望与方差 - 8}
\vspace{6em}

% --- 题目 ---
\problem[P214-10 (21-1,3)]{在区间 $\displaystyle (0, 2)$ 上随机取一点, 将该区间分成两段, 较短一段的长度记为 $\displaystyle X$, 较长一段的长度记为 $\displaystyle Y$. 令 $\displaystyle Z = \frac{Y}{X}$.
\begin{enumerate}
\item[(1)] 求 $\displaystyle X$ 的概率密度;
\item[(2)] 求 $\displaystyle Z$ 的概率密度;
\item[(3)] 求 $\displaystyle E\left(\frac{X}{Y}\right)$.
\end{enumerate}}
\begin{note}
  主要错的是 (3)
\end{note}
\ansat{363;【十年真题】 - 考点一:随机变量的数学期望与方差 - 10}

% --- 题目 ---
\problem[P215-1 (25-1)]{设二维随机变量 $\displaystyle (X, Y)$ 服从正态分布 $\displaystyle N(0, 0; 1, 1; \rho)$, 其中 $\displaystyle \rho \in (-1, 1)$. 若 $\displaystyle a, b$ 为满足 $\displaystyle a^2 + b^2 = 1$ 的任意实数, 则 $\displaystyle D(aX + bY)$ 的最大值为~(~\quad~)}
\begin{tasks}(4)
  \task $\displaystyle 1.$
  \task $\displaystyle 2.$
  \task $\displaystyle 1 + |\rho|.$
  \task $\displaystyle 1 + \rho^2.$
\end{tasks}
\ansat{363;【十年真题】 - 考点二:随机变量的协方差与相关系数 - 1}

% --- 题目 ---
\problem[P215-5 (22-1)]{设随机变量 $\displaystyle X \sim N(0, 1)$, 在 $\displaystyle X=x$ 的条件下随机变量 $\displaystyle Y \sim N(x, 1)$, 则 $\displaystyle X$ 与 $\displaystyle Y$ 的相关系数为~(~\quad~)}
\begin{tasks}(4)
  \task $\displaystyle \frac{1}{4}.$
  \task $\displaystyle \frac{1}{2}.$
  \task $\displaystyle \frac{\sqrt{3}}{3}.$
  \task $\displaystyle \frac{\sqrt{2}}{2}.$
\end{tasks}
\ansat{364;【十年真题】 - 考点二:随机变量的协方差与相关系数 - 5}

% --- 题目 ---
\problem[P215-8 (20-3)]{设随机变量 $\displaystyle (X, Y)$ 服从二维正态分布 $\displaystyle N\left(0, 0; 1, 4; -\frac{1}{2}\right)$, 则下列随机变量中服从标准正态分布且与 $\displaystyle X$ 独立的是~(~\quad~)}
\begin{tasks}(4)
  \task $\displaystyle \frac{\sqrt{5}}{5}(X+Y).$
  \task $\displaystyle \frac{\sqrt{5}}{5}(X-Y).$
  \task $\displaystyle \frac{\sqrt{3}}{3}(X+Y).$
  \task $\displaystyle \frac{\sqrt{3}}{3}(X-Y).$
\end{tasks}
\ansat{364;【十年真题】 - 考点二:随机变量的协方差与相关系数 - 8}
\vspace{2em}

% --- 题目 ---
\problem[P215-10 (23-3)]{设随机变量 $\displaystyle X$ 与 $\displaystyle Y$ 相互独立, 且 $\displaystyle X \sim B(1, p)$, $\displaystyle Y \sim B(2, p), p \in (0, 1)$, 则 $\displaystyle X+Y$ 与 $\displaystyle X-Y$ 的相关系数为 \underline{\hspace{4em}}.}
\ansat{364;【十年真题】 - 考点二:随机变量的协方差与相关系数 - 10}

% --- 题目 ---
\problem[P215-12 (20-1)]{设 $\displaystyle X$ 服从区间 $\displaystyle \left(-\frac{\pi}{2}, \frac{\pi}{2}\right)$ 上的均匀分布, $\displaystyle Y = \sin X$, 则 $\displaystyle \mathrm{Cov}(X, Y) = \underline{\hspace{4em}}. $}
\ansat{365;【十年真题】 - 考点二:随机变量的协方差与相关系数 - 12}

% --- 题目 ---
\problem[P215-13 (23-1)]{设二维随机变量 $\displaystyle (X, Y)$ 的概率密度为
\[ f(x, y) = \begin{cases} \displaystyle \frac{2}{\pi}(x^2 + y^2), & \displaystyle x^2 + y^2 \le 1, \\ 0, & \text{其他}. \end{cases} \]
\begin{enumerate}
\item[(1)] 求 $\displaystyle X$ 与 $\displaystyle Y$ 的协方差;
\item[(2)] $\displaystyle X$ 与 $\displaystyle Y$ 是否相互独立?
\item[(3)] 求 $\displaystyle Z = X^2 + Y^2$ 的概率密度.
\end{enumerate}}
\ansat{365;【十年真题】 - 考点二:随机变量的协方差与相关系数 - 13}

% --- 题目 ---
\problem[P216-15 (19-1,3)]{设随机变量 $\displaystyle X$ 与 $\displaystyle Y$ 相互独立, $\displaystyle X$ 服从参数为 1 的指数分布, $\displaystyle Y$ 的概率分布为
\[ P\{Y = -1\} = p, \quad P\{Y = 1\} = 1-p \quad (0 < p < 1). \]
令 $\displaystyle Z = XY$.
\begin{enumerate}
\item[(1)] 求 $\displaystyle Z$ 的概率密度;
\item[(2)] $\displaystyle p$ 为何值时, $\displaystyle X$ 与 $\displaystyle Z$ 不相关?
\item[(3)] $\displaystyle X$ 与 $\displaystyle Z$ 是否相互独立?
\end{enumerate}}
\ansat{365;【十年真题】 - 考点二:随机变量的协方差与相关系数 - 15}
\vspace{6em}

% --- 题目 ---
\problem[P216-16 (18-1,3)]{设随机变量 $\displaystyle X$ 与 $\displaystyle Y$ 相互独立, $\displaystyle X$ 的概率分布为
\[ P\{X=1\} = P\{X=-1\} = \frac{1}{2}, \]
$\displaystyle Y$ 服从参数为 $\displaystyle \lambda$ 的泊松分布. 令 $\displaystyle Z=XY$.
\begin{enumerate}
\item[(1)] 求 $\displaystyle \mathrm{Cov}(X, Z)$;
\item[(2)] 求 $\displaystyle Z$ 的概率分布.
\end{enumerate}}
\ansat{365;【十年真题】 - 考点二:随机变量的协方差与相关系数 - 16}

% --- 题目 ---
\problem[P216-例2]{设随机变量 $\displaystyle X \sim N(0, 1), Y \sim N(1, 4)$, 且 $\displaystyle U = aX + Y, V = aX - Y (a > 0)$. 若 $\displaystyle U, V$ 不相关, 则 $\displaystyle a = \underline{\hspace{4em}}.$}
\ansat{219;【方法探究】 - 考点二:随机变量的协方差与相关系数 - 例2}

% --- 题目 ---
\problem[P219-2 (09-1)]{设随机变量 $\displaystyle X$ 的分布函数为 $\displaystyle F(x) = 0.3\Phi(x) + 0.7\Phi\left(\frac{x-1}{2}\right)$, 其中 $\displaystyle \Phi(x)$ 为标准正态分布函数, 则 $\displaystyle EX $= ( \quad )}
\begin{tasks}(4)
  \task 0.
  \task 0.3.
  \task 0.7.
  \task 1.
\end{tasks}
\ansat{366;【真题精选】 - 考点一:随机变量的数学期望与方差 - 2}

% --- 题目 ---
\problem[P219-3 (13-3)]{设随机变量 $\displaystyle X$ 服从标准正态分布 $\displaystyle N(0, 1)$, 则 $\displaystyle E(X\mathrm{e}^{2X}) = \underline{\hspace{4em}}.$}
\ansat{366;【真题精选】 - 考点一:随机变量的数学期望与方差 - 3}

% --- 题目 ---
\problem[P219-9 (99-3)]{设随机变量 $\displaystyle X_{ij} \ (i, j = 1, 2, \dots, n; \ n \ge 2)$ 独立同分布, $\displaystyle EX_{ij} = 2$, 则行列式
\[ Y = \begin{vmatrix} X_{11} & X_{12} & \dots & X_{1n} \\ X_{21} & X_{22} & \dots & X_{2n} \\ \vdots & \vdots & & \vdots \\ X_{n1} & X_{n2} & \dots & X_{nn} \end{vmatrix} \]
的数学期望 $\displaystyle EY = \underline{\hspace{4em}}.$}
\ansat{366;【真题精选】 - 考点一:随机变量的数学期望与方差 - 9}

% --- 题目 ---
\problem[P219-12 (15-1,3)]{设随机变量 $\displaystyle X$ 的概率密度为
\[ f(x) = \begin{cases} 2^{-x} \ln 2, & x > 0, \\ 0, & x \le 0. \end{cases} \]
对 $\displaystyle X$ 进行独立重复的观测, 直到第 2 个大于 3 的观测值出现时停止, 记 $\displaystyle Y$ 为观测次数.
\begin{enumerate}
\item[(1)] 求 $\displaystyle Y$ 的概率分布;
\item[(2)] 求 $\displaystyle EY$.
\end{enumerate}}
\ansat{366;【真题精选】 - 考点一:随机变量的数学期望与方差 - 12}

% --- 题目 ---
\problem[P220-14 (03-1)]{已知甲、乙两箱中装有同种产品, 其中甲箱中装有 3 件合格品和 3 件次品, 乙箱中仅装有 3 件合格品. 从甲箱中任取 3 件产品放入乙箱后, 求:
\begin{enumerate}
\item[(1)] 乙箱中次品件数 $\displaystyle X$ 的数学期望;
\item[(2)] 从乙箱中任取一件产品是次品的概率.
\end{enumerate}}
\ansat{366;【真题精选】 - 考点一:随机变量的数学期望与方差 - 14}

% --- 题目 ---
\problem[P221-5 (06-3)]{设随机变量 $\displaystyle X$ 的概率密度为
\[ f_X(x) = \begin{cases} \frac{1}{2}, & -1 < x < 0, \\ \frac{1}{4}, & 0 < x < 2, \\ 0, & \text{其他.} \end{cases} \]
令 $\displaystyle Y = X^2$, $\displaystyle F(x, y)$ 为二维随机变量 $\displaystyle (X, Y)$ 的分布函数. 求:
\begin{enumerate}
\item[(1)] $\displaystyle Y$ 的概率密度 $\displaystyle f_Y(y)$;
\item[(2)] $\displaystyle \mathrm{Cov}(X, Y)$;
\item[(3)] $\displaystyle F\left(-\frac{1}{2}, 4\right)$.
\end{enumerate}}
\ansat{367;【真题精选】 - 考点二:随机变量的协方差与相关系数 - 5}