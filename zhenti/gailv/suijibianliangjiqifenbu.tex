% --- 题目 ---
\problem[P200-2 (16-1)]{设随机变量 $\displaystyle X \sim N(\mu, \sigma^2) \ (\sigma > 0)$, 记 $\displaystyle p = P\{X \le \mu + \sigma^2\}$, 则 ( \quad )}
\begin{tasks}(2)
  \task $\displaystyle p$ 随着 $\displaystyle \mu$ 的增加而增加.
  \task $\displaystyle p$ 随着 $\displaystyle \sigma$ 的增加而增加.
  \task $\displaystyle p$ 随着 $\displaystyle \mu$ 的增加而减少.
  \task $\displaystyle p$ 随着 $\displaystyle \sigma$ 的增加而减少.
\end{tasks}
\ansat{355;【十年真题】 - 考点一:随机变量的分布 - 2}

% --- 题目 ---
\problem[P200-3 (24-1,3)]{设随机试验每次成功的概率为 $\displaystyle p$, 现进行 3 次独立重复试验, 在至少成功 1 次的条件下 3 次试验全部成功的概率为 $\frac{4}{13}$, 则 $\displaystyle p = \underline{\hspace{4em}}. $}
\ansat{355;【十年真题】 - 考点一:随机变量的分布 - 3}

% --- 题目 ---
\problem[P200-1 (23-3-局部)]{设随机变量 $\displaystyle X$ 的概率密度为 $\displaystyle f(x) = \frac{\mathrm{e}^x}{(1+\mathrm{e}^x)^2}, -\infty < x < +\infty$, 令 $\displaystyle Y = \mathrm{e}^X$.}
\begin{enumerate}
\item[(1)] 求 $\displaystyle X$ 的分布函数;
\item[(2)] 求 $\displaystyle Y$ 的概率密度.
\end{enumerate}
\ansat{355;【十年真题】 - 考点二:随机变量的函数的分布 - 1}

% --- 题目 ---
\problem[P202-例3 (90-1)]{已知随机变量 $\displaystyle X$ 的概率密度函数 $\displaystyle f(x) = \frac{1}{2}\mathrm{e}^{-|x|}, -\infty < x < +\infty$, 则 $\displaystyle X$ 的分布函数 $\displaystyle F(x) = \underline{\hspace{4em}}. $}
\ansat{202;【方法探究】 - 考点一:随机变量的分布 - 例3}

% --- 题目 ---
\problem[P203-4 (02-1)]{设 $\displaystyle X_1$ 和 $\displaystyle X_2$ 是相互独立的连续型随机变量, 它们的密度函数分别为 $\displaystyle f_1(x)$ 和 $\displaystyle f_2(x)$, 分布函数分别为 $\displaystyle F_1(x)$ 和 $\displaystyle F_2(x)$, 则 ( \quad )}
\begin{tasks}(1)
  \task $\displaystyle f_1(x) + f_2(x)$ 必为某一随机变量的概率密度.
  \task $\displaystyle f_1(x) f_2(x)$ 必为某一随机变量的概率密度.
  \task $\displaystyle F_1(x) + F_2(x)$ 必为某一随机变量的分布函数.
  \task $\displaystyle F_1(x) F_2(x)$ 必为某一随机变量的分布函数.
\end{tasks}
\ansat{356;【真题精选】 - 考点一:随机变量的分布 - 4}

% --- 题目 ---
\problem[P204-7 (93-3)]{设随机变量 $\displaystyle X$ 的概率密度为 $\displaystyle \varphi(x)$, 且 $\displaystyle \varphi(-x) = \varphi(x)$. $\displaystyle F(x)$ 为 $\displaystyle X$ 的分布函数, 则对任意实数 $\displaystyle a$, 有 ( \quad )}
\begin{tasks}(2)
  \task $\displaystyle F(-a) = 1 - \int_{0}^{a} \varphi(x) \ \mathrm{d}x.$
  \task $\displaystyle F(-a) = \frac{1}{2} - \int_{0}^{a} \varphi(x) \ \mathrm{d}x.$
  \task $\displaystyle F(-a) = F(a).$
  \task $\displaystyle F(-a) = 2F(a) - 1.$
\end{tasks}
\ansat{356;【真题精选】 - 考点一:随机变量的分布 - 7}

% --- 题目 ---
\problem[P204-11 (89-4)]{设随机变量 $\displaystyle X$ 的分布函数为
\[ F(x) = \begin{cases} 0, & \displaystyle x < 0, \\ \displaystyle A \sin x, & \displaystyle 0 \le x \le \frac{\pi}{2}, \\ 1, & \displaystyle x > \frac{\pi}{2}, \end{cases} \]
则 $\displaystyle P\left\{|X| < \frac{\pi}{6}\right\} = \underline{\hspace{4em}}. $}
\ansat{356;【真题精选】 - 考点一:随机变量的分布 - 11}

% --- 题目 ---
\problem[P204-2 (03-3)]{设随机变量 $\displaystyle X$ 的概率密度为
$$ f(x) = \begin{cases} \frac{1}{3\sqrt[3]{x^2}}, & x \in [1, 8], \\ 0, & \text{其他}, \end{cases} $$
$\displaystyle F(x)$ 是 $\displaystyle X$ 的分布函数, 则随机变量 $\displaystyle Y=F(X)$ 的分布函数为 \underline{\hspace{4em}}.}
\ansat{357;【真题精选】 - 考点二:随机变量的函数的分布 - 2}

% --- 题目 ---
\problem[P204-3 (13-1)]{设随机变量 $\displaystyle X$ 的概率密度为 $\displaystyle f(x) = \begin{cases} \frac{1}{9}x^2, & 0 < x < 3, \\ 0, & \text{其他}. \end{cases}$, 
\newline
令随机变量 $\displaystyle Y = \begin{cases} 2, & X \le 1, \\ X, & 1 < X < 2, \\ 1, & X \ge 2. \end{cases}.$
\begin{enumerate}
\item[(1)] 求 \(Y\) 的分布函数;
\item[(2)] 求概率 \(P\left\{ X \leq Y \right\}\).
\end{enumerate}}
\ansat{357;【真题精选】 - 考点二:随机变量的函数的分布 - 3}
