% --- 题目 ---
\problem[P196-7 (25-1)]{设 $\displaystyle A, B$ 为两个随机事件, 且 $\displaystyle A$ 与 $\displaystyle B$ 相互独立. 已知 $\displaystyle P(A)=2P(B)$, $\displaystyle P(A \cup B)=\frac{5}{8}$, 则在事件 $\displaystyle A, B$ 至少有一个发生的条件下, $\displaystyle A, B$ 中恰有一个发生的概率为 \underline{\hspace{4em}}.}
\ansat{353;【十年真题】 - 考点一:概率的五大公式 - 7}

% --- 题目 ---
\problem[P196-11 (18-3)]{设随机事件 $\displaystyle A, B, C$ 相互独立, 且}
\[ P(A) = P(B) = P(C) = \frac{1}{2}, \]
{则 $\displaystyle P(AC|A \cup B) = \underline{\hspace{4em}}.$}
\ansat{354;【十年真题】 - 考点一:概率的五大公式 - 11}

% --- 题目 ---
\problem[P198-例(1)]{袋中装有 $\displaystyle 2$ 个红球, $\displaystyle 3$ 个黄球, $\displaystyle 1$ 个蓝球. 现有放回地从袋中取 $\displaystyle 3$ 次球, 每次取 $\displaystyle 1$ 个球, 则恰有 $\displaystyle 1$ 次取到蓝球 的概率为 \underline{\hspace{4em}}.}
\ansat{198;【方法探究】 - 考点二:古典概型与几何概型 - 例 (1)}

% --- 题目 ---
\problem[P198-1 (15-1,3)]{若 $\displaystyle A, B$ 为任意两个随机事件, 则~(~\quad~)}
\begin{tasks}(2)
  \task $\displaystyle P(AB) \le P(A)P(B).$
  \task $\displaystyle P(AB) \ge P(A)P(B).$
  \task $\displaystyle P(AB) \le \frac{P(A)+P(B)}{2}.$
  \task $\displaystyle P(AB) \ge \frac{P(A)+P(B)}{2}.$
\end{tasks}
\ansat{354;【真题精选】 - 考点一:概率的五大公式 - 1}

% --- 题目 ---
\problem[P198-11 (97-1)]{袋中有 $\displaystyle 50$ 个乒乓球, 其中 $\displaystyle 20$ 个是黄球, $\displaystyle 30$ 个是白球, 今有两人依次随机地从袋中各取一球, 取后不放回, 则第二个人取得黄球的概率是 \underline{\hspace{4em}}.}
\ansat{354;【真题精选】 - 考点一:概率的五大公式 - 11}

% --- 题目 ---
\problem[P199-14 (89-1)]{甲、乙两人独立地对同一目标射击一次, 其命中率分别为 $\displaystyle 0.6$ 和 $\displaystyle 0.5$. 现已知目标被命中, 则它是甲射中的概率为 \underline{\hspace{4em}}.}
\ansat{355;【真题精选】 - 考点一:概率的五大公式 - 14}
