% --- 题目 ---
\problem[P2-1 (24-2)]{已知数列 $\displaystyle \{a_n\}$ ($\displaystyle a_n \neq 0$). 若 $\displaystyle \{a_n\}$ 发散, 则 ( \quad )}
\begin{tasks}(2)
  \task $\displaystyle \left\{a_n + \frac{1}{a_n}\right\}$ 发散.
  \task $\displaystyle \left\{a_n - \frac{1}{a_n}\right\}$ 发散.
  \task $\displaystyle \left\{\mathrm{e}^{a_n} + \frac{1}{\mathrm{e}^{a_n}}\right\}$ 发散.
  \task $\displaystyle \left\{\mathrm{e}^{a_n} - \frac{1}{\mathrm{e}^{a_n}}\right\}$ 发散.
\end{tasks}
\ansat{238;【十年真题】 - 考点:极限的概念与性质 - 1}

% --- 题目 ---
\problem[P2-2 (22-1,2)]{设数列 $\displaystyle \{x_n\}$ 满足 $\displaystyle -\frac{\pi}{2} \leqslant x_n \leqslant \frac{\pi}{2}$, 则 ( \quad )}
\begin{tasks}(1)
  \task 当 $\displaystyle \lim_{n \to \infty} \cos(\sin x_n)$ 存在时, $\displaystyle \lim_{n \to \infty} x_n$ 存在.
  \task 当 $\displaystyle \lim_{n \to \infty} \sin(\cos x_n)$ 存在时, $\displaystyle \lim_{n \to \infty} x_n$ 存在.
  \task 当 $\displaystyle \lim_{n \to \infty} \cos(\sin x_n)$ 存在时, $\displaystyle \lim_{n \to \infty} \sin x_n$ 存在, 但 $\displaystyle \lim_{n \to \infty} x_n$ 不一定存在.
  \task 当 $\displaystyle \lim_{n \to \infty} \sin(\cos x_n)$ 存在时, $\displaystyle \lim_{n \to \infty} \cos x_n$ 存在, 但 $\displaystyle \lim_{n \to \infty} x_n$ 不一定存在.
\end{tasks}
\ansat{238;【十年真题】 - 考点:极限的概念与性质 - 2}

% --- 题目 ---
\problem[P4-4 (99-2)]{对任意给定的 $\displaystyle \varepsilon \in (0, 1)$,总存在正整数 $\displaystyle N$,当 $\displaystyle n \ge N$ 时,恒有 $\displaystyle |x_n - a| \le 2\varepsilon$" 是数列 $\displaystyle \{x_n\}$ 收敛于 $\displaystyle a$ 的 ( \quad )}
\begin{tasks}(2)
  \task 充分条件但非必要条件.
  \task 必要条件但非充分条件.
  \task 充分必要条件.
  \task 既非充分条件又非必要条件.
\end{tasks}
\ansat{238;【真题精选】 - 考点:极限的概念与性质 - 4}

% --- 题目 ---
\problem[P4-2 (23-3)]{$\displaystyle \lim_{x \to \infty} x^2 \left(2 - x \sin \frac{1}{x} - \cos \frac{1}{x}\right) = \underline{\hspace{3em}}.$}
\ansat{238;【十年真题】 - 考点一:函数极限的计算 - 2}

% --- 题目 ---
\problem[P4-3 (22-2,3)]{$\displaystyle \lim_{x \to 0} \left(\frac{1+\mathrm{e}^x}{2}\right)^{\cot x} = \underline{\hspace{3em}}.$}
\ansat{238;【十年真题】 - 考点一:函数极限的计算 - 3}

% --- 题目 ---
\problem[P4-7 (16-2,3)]{求极限 $\displaystyle \lim_{x \to 0} (\cos 2x + 2x \sin x)^{\frac{1}{x^4}}$.}
\ansat{239;【十年真题】 - 考点一:函数极限的计算 - 7}

% --- 题目 ---
\problem[P4-2 (25-2)]{$\displaystyle \lim_{n \to \infty} \frac{1}{n^2} \left[ \ln \frac{1}{n} + 2\ln \frac{2}{n} + \dots + (n-1)\ln \frac{n-1}{n} \right] = \underline{\hspace{3em}}.$}
\ansat{239;【十年真题】 - 考点二:数列极限的计算 - 2}

% --- 题目 ---
\problem[P4-5 (19-1,3)]{设 $\displaystyle a_n = \int_{0}^{1} x^n \sqrt{1 - x^2} \mathrm{d}x \ (n = 0, 1, 2, \dots)$.}
\begin{enumerate}
\item[(1)] 证明: 数列 $\displaystyle \{a_n\}$ 单调减少, 且 $\displaystyle a_n = \frac{n-1}{n+2} a_{n-2} \ (n=2, 3, \dots)$;
\item[(2)] 求 $\displaystyle \lim_{n \to \infty} \frac{a_n}{a_{n-1}}$.
\end{enumerate}
\ansat{239;【十年真题】 - 考点二:数列极限的计算 - 5}

% --- 题目 ---
\problem[P4-6 (18-1,2,3)]{设数列 $\displaystyle \{x_n\}$ 满足:$\displaystyle x_1 > 0, \ x_n \mathrm{e}^{x_{n+1}} = \mathrm{e}^{x_n} - 1 \ (n = 1, 2, \dots)$. 证明 $\displaystyle \{x_n\}$ 收敛, 并求 $\displaystyle \lim_{n \to \infty} x_n$.}
\ansat{240;【十年真题】 - 考点二:数列极限的计算 - 6}

% --- 题目 ---
\problem[P6-例1 (2)]{$\displaystyle \lim_{x \to 0} \frac{\sqrt{1+x} - \sqrt{1+\tan x}}{x \tan^2 x} = \underline{\hspace{3em}}.$}
\ansat{6;【方法探究】 - 考点一:函数极限的计算 - 例1 (2)}

% --- 题目 ---
\problem[P7-变式 1.1 (97-2)]{求极限 $\displaystyle \lim_{x \to -\infty} \frac{\sqrt{4x^2+x-1}+x+1}{\sqrt{x^2+\sin x}}$.}
\ansat{240;【方法探究】 - 考点一:函数极限的计算 - 变式 1.1}

% --- 题目 ---
\problem[P8-变式 3 (98-1)]{求极限 $$\lim_{n \to \infty} \left( \frac{\sin \frac{\pi}{n}}{n+1} + \frac{\sin \frac{2\pi}{n}}{n + \frac{1}{2}} + \dots + \frac{\sin \pi}{n + \frac{1}{n}} \right).$$}
\ansat{240;【方法探究】 - 考点二:数列极限的计算 - 变式 3}

% --- 题目 ---
\probtagged[P8-变式 4.1 (96-1)]{设 $\displaystyle x_1 = 10, \ x_{n+1} = \sqrt{6+x_n} \ (n = 1, 2, \dots)$. 试证数列 $\displaystyle \{x_n\}$ 极限存在, 并求此极限.}
\ansat{241;【方法探究】 - 考点二:数列极限的计算 - 变式 4.1}

% --- 题目 ---
\problem[P8-变式 4.2 (11-1, 2)]{(1) 证明:对任意的正整数 $\displaystyle n$, 都有 $\displaystyle \frac{1}{n+1} < \ln\left(1+\frac{1}{n}\right) < \frac{1}{n}$ 成立;
\newline
(2) 设 $\displaystyle a_n = 1 + \frac{1}{2} + \dots + \frac{1}{n} - \ln n (n = 1, 2, \dots)$, 证明数列 $\displaystyle \{a_n\}$ 收敛.}
\begin{note}
  主要错的是 (2)
\end{note}
\ansat{241;【方法探究】 - 考点二:数列极限的计算 - 变式 4.2}

% --- 题目 ---
\problem[P9-8 (97-1)]{$\displaystyle \lim_{x \to 0} \frac{3\sin x + x^2 \cos \frac{1}{x}}{(1 + \cos x)\ln(1 + x)} = \underline{\hspace{4em}}. $}
\ansat{241;【真题精选】 - 考点一:函数极限的计算 - 8}

% --- 题目 ---
\problem[P9-15 (08-3)]{计算 \[ \displaystyle \lim_{x \to 0} \frac{1}{x^2} \ln \frac{\sin x}{x}. \]}
\ansat{242;【真题精选】 - 考点一:函数极限的计算 - 15}

% --- 题目 ---
\problem[P10-1 (12-2)]{设 $\displaystyle a_n > 0 \ $ ($\displaystyle n=1,2,\dots$), $\displaystyle S_n = a_1 + a_2 + \dots + a_n$, 则数列 $\displaystyle \{S_n\}$ 有界是数列 $\displaystyle \{a_n\}$ 收敛的( \quad )}
\begin{tasks}(2)
\task 充分必要条件.
\task 充分非必要条件.
\task 必要非充分条件.
\task 既非充分也非必要条件.
\end{tasks}
\ansat{243;【真题精选】 - 考点二:数列极限的计算 - 1}

% --- 题目 ---
\problem[P10-2 (04-2)]{$\displaystyle \lim_{n \to \infty} \ln \sqrt[n]{ \left(1+\frac{1}{n}\right)^2 \left(1+\frac{2}{n}\right)^2 \cdots \left(1+\frac{n}{n}\right)^2 }$ 等于 ( \quad )}
\begin{tasks}(2)
 \task $\displaystyle \int_{1}^{2} \ln^2 x \, \mathrm{d}x.$
 \task $\displaystyle 2 \int_{1}^{2} \ln x \, \mathrm{d}x.$
 \task $\displaystyle 2 \int_{1}^{2} \ln(1+x) \, \mathrm{d}x.$
 \task $\displaystyle \int_{1}^{2} \ln^2(1+x) \, \mathrm{d}x.$
\end{tasks}
\ansat{243;【真题精选】 - 考点二:数列极限的计算 - 2}

% --- 题目 ---
\problem[P10-4 (02-2)]{$\displaystyle \lim_{n \to \infty} \frac{1}{n} \left[ \sqrt{1+\cos \frac{\pi}{n}} + \sqrt{1+\cos \frac{2\pi}{n}} + \ldots + \sqrt{1+\cos \frac{n\pi}{n}} \right] = \underline{\hspace{4em}}. $}
\ansat{243;【真题精选】 - 考点二:数列极限的计算 - 4}

% --- 题目 ---
\problem[P10-6 (13-2)]{设函数 $\displaystyle f(x) = \ln x + \frac{1}{x}$.
\begin{enumerate}
\item[(1)] 求 $\displaystyle f(x)$ 的最小值;
\item[(2)] 设数列 $\displaystyle \{x_n\}$ 满足 $\displaystyle \ln x_n + \frac{1}{x_{n+1}} < 1$. 证明 $\displaystyle \lim_{n \to \infty} x_n$ 存在, 并求此极限.
\end{enumerate}}
\begin{note}
  主要错的是 (2)
\end{note}
\ansat{243;【真题精选】 - 考点二:数列极限的计算 - 6}

% --- 题目 ---
\problem[P10-8 (99-2)]{设 $\displaystyle f(x)$ 是区间 $\displaystyle [0, +\infty)$ 上单调减少且非负的连续函数.
\[ a_n = \displaystyle \sum_{k=1}^{n} f(k) - \int_{1}^{n} f(x) \ \mathrm{d}x \ (n = 1, 2, \dots), \]
证明数列 $\displaystyle \{a_n\}$ 的极限存在.}
\ansat{243;【真题精选】 - 考点二:数列极限的计算 - 8}

% --- 题目 ---
\problem[P11-3 (23-2)]{已知数列 $\displaystyle \{x_n\}, \{y_n\}$ 满足 $\displaystyle x_1 = y_1 = \frac{1}{2}, \ x_{n+1} = \sin x_n, \ y_{n+1} = y_n^2 \ (n=1, 2, \dots)$, 则当 $\displaystyle n \to \infty$ 时 ( \quad )}
\begin{tasks}(2)
\task $\displaystyle x_n$ 是 $\displaystyle y_n$ 的高阶无穷小.
\task $\displaystyle y_n$ 是 $\displaystyle x_n$ 的高阶无穷小.
\task $\displaystyle x_n$ 与 $\displaystyle y_n$ 是等价无穷小.
\task $\displaystyle x_n$ 与 $\displaystyle y_n$ 是同阶但不等价的无穷小.
\end{tasks}
\ansat{244;【十年真题】 - 考点一:无穷小的比较 - 3}

% --- 题目 ---
\problem[P11-6 (20-2)]{求曲线 $\displaystyle y = \frac{x^{1+x}}{(1+x)^x} \ (x > 0)$ 的斜渐近线方程.}
\ansat{244;【十年真题】 - 考点二:平面曲线的渐近线 - 6}

% --- 题目 ---
\problem[P14-2 (12-1,2,3)]{曲线 $\displaystyle y = \frac{x^2+x}{x^2-1}$ 渐近线的条数为 ( \quad )}
\begin{tasks}(4)
\task 0.
\task 1.
\task 2.
\task 3.
\end{tasks}
\ansat{245;【真题精选】 - 考点二:平面曲线的渐近线 - 2}

% --- 题目 ---
\problem[P14-3 (07-1,3)]{曲线 $\displaystyle y = \frac{1}{x} + \ln(1 + \mathrm{e}^x)$ 渐近线的条数为 ( \quad )}
\begin{tasks}(4)
\task 0.
\task 1.
\task 2.
\task 3.
\end{tasks}
\ansat{245;【真题精选】 - 考点二:平面曲线的渐近线 - 3}

% --- 题目 ---
\problem[P14-8 (00-3)]{求函数 $\displaystyle f(x)=(x-1)\mathrm{e}^{\frac{\pi}{2} + \arctan x}$ 图形的渐近线.}
\ansat{246;【真题精选】 - 考点二:平面曲线的渐近线 - 8}
\vspace{6em}

% --- 题目 ---
\problem[P14-4 (07-2)]{函数 $\displaystyle f(x) = \frac{(\mathrm{e}^{\frac{1}{x}} + \mathrm{e})\tan x}{x(\mathrm{e}^{\frac{1}{x}} - \mathrm{e})}$ 在区间 $\displaystyle [-\pi, \pi]$ 上的第一类间断点是 $\displaystyle x= ( \quad )$
\begin{tasks}(2)
\task $\displaystyle 0$.
\task $\displaystyle 1$.
\task $\displaystyle -\frac{\pi}{2}$.
\task $\displaystyle \frac{\pi}{2}$.
\end{tasks}}
\ansat{246;【真题精选】 - 考点三:函数的连续性与间断点 - 4}

% --- 题目 ---
\problem[P14-6 (95-2)]{设 $\displaystyle f(x)$ 和 $\displaystyle \varphi(x)$ 在 $\displaystyle (-\infty, +\infty)$ 上有定义, $\displaystyle f(x)$ 为连续函数, 且 $\displaystyle f(x) \neq 0$, $\displaystyle \varphi(x)$ 有间断点, 则 ( \quad )}
\begin{tasks}(2)
\task $\displaystyle \varphi[f(x)]$ 必有间断点.
\task $\displaystyle [\varphi(x)]^2$ 必有间断点.
\task $\displaystyle f[\varphi(x)]$ 必有间断点.
\task $\displaystyle \frac{\varphi(x)}{f(x)}$ 必有间断点.
\end{tasks}
\ansat{246;【真题精选】 - 考点三:函数的连续性与间断点 - 6}

% --- 题目 ---
\problem[P14-9 (01-2)]{求极限 $\displaystyle \lim_{t \to x} \left(\frac{\sin t}{\sin x}\right)^{\frac{x}{\sin t - \sin x}}$, 记此极限为 $\displaystyle f(x)$, 求函数 $\displaystyle f(x)$ 的间断点并指出其类型.}
\ansat{247;【真题精选】 - 考点三:函数的连续性与间断点 - 9}

% --- 题目 ---
\problem[P15-1 (20-3)]{设 $\displaystyle \lim_{x \to a} \frac{f(x)-a}{x-a} = b$, 则 $\displaystyle \lim_{x \to a} \frac{\sin f(x) - \sin a}{x-a} = ( \quad )$
\begin{tasks}(2)
\task $\displaystyle b \sin a$.
\task $\displaystyle b \cos a$.
\task $\displaystyle b \sin f(a)$.
\task $\displaystyle b \cos f(a)$.
\end{tasks}}
\ansat{247;【十年真题】 - 考点一:已知极限求另一极限 - 1}

% --- 题目 ---
\problem[P15-2 (16-3)]{已知函数 $\displaystyle f(x)$ 满足 $\displaystyle \lim_{x \to 0} \frac{\sqrt{1+f(x)\sin 2x}-1}{\mathrm{e}^{3x}-1} = 2$, 则 $\displaystyle \lim_{x \to 0} f(x) = \underline{\hspace{4em}}.$}
\ansat{247;【十年真题】 - 考点一:已知极限求另一极限 - 2}
