% --- 题目 ---
\problem[P2-1 (24-2)]{已知数列 $\displaystyle \{a_n\}$ ($\displaystyle a_n \neq 0$). 若 $\displaystyle \{a_n\}$ 发散, 则 ( \quad )}
\begin{tasks}(2)
  \task $\displaystyle \left\{a_n + \frac{1}{a_n}\right\}$ 发散.
  \task $\displaystyle \left\{a_n - \frac{1}{a_n}\right\}$ 发散.
  \task $\displaystyle \left\{\mathrm{e}^{a_n} + \frac{1}{\mathrm{e}^{a_n}}\right\}$ 发散.
  \task $\displaystyle \left\{\mathrm{e}^{a_n} - \frac{1}{\mathrm{e}^{a_n}}\right\}$ 发散.
\end{tasks}
\ansat{238;【十年真题】 - 考点:极限的概念与性质 - 1}

% --- 题目 ---
\problem[P2-2 (22-1,2)]{设数列 $\displaystyle \{x_n\}$ 满足 $\displaystyle -\frac{\pi}{2} \leqslant x_n \leqslant \frac{\pi}{2}$, 则 ( \quad )}
\begin{tasks}(1)
  \task 当 $\displaystyle \lim_{n \to \infty} \cos(\sin x_n)$ 存在时, $\lim_{n \to \infty} x_n$ 存在.
  \task 当 $\displaystyle \lim_{n \to \infty} \sin(\cos x_n)$ 存在时, $\lim_{n \to \infty} x_n$ 存在.
  \task 当 $\displaystyle \lim_{n \to \infty} \cos(\sin x_n)$ 存在时, $\lim_{n \to \infty} \sin x_n$ 存在, 但 $\lim_{n \to \infty} x_n$ 不一定存在.
  \task 当 $\displaystyle \lim_{n \to \infty} \sin(\cos x_n)$ 存在时, $\lim_{n \to \infty} \cos x_n$ 存在, 但 $\displaystyle \lim_{n \to \infty} x_n$ 不一定存在.
\end{tasks}
\ansat{238;【十年真题】 - 考点:极限的概念与性质 - 2}

% --- 题目 ---
\problem[P4-4 (99-2)]{对任意给定的 $\displaystyle \varepsilon \in (0, 1)$,总存在正整数 $\displaystyle N$,当 $\displaystyle n \ge N$ 时,恒有 $\displaystyle |x_n - a| \le 2\varepsilon$" 是数列 $\displaystyle \{x_n\}$ 收敛于 $\displaystyle a$ 的 ( \quad )}
\begin{tasks}(2)
  \task 充分条件但非必要条件.
  \task 必要条件但非充分条件.
  \task 充分必要条件.
  \task 既非充分条件又非必要条件.
\end{tasks}
\ansat{238;【真题精选】 - 考点:极限的概念与性质 - 4}

% --- 题目 ---
\problem[P4-2 (23-3)]{$\displaystyle \lim_{x \to \infty} x^2 \left(2 - x \sin \frac{1}{x} - \cos \frac{1}{x}\right) = \underline{\hspace{3em}}.$}
\ansat{238;【十年真题】 - 考点一:函数极限的计算 - 2}

% --- 题目 ---
\problem[P4-3 (22-2,3)]{$\displaystyle \lim_{x \to 0} \left(\frac{1+\mathrm{e}^x}{2}\right)^{\cot x} = \underline{\hspace{3em}}.$}
\ansat{238;【十年真题】 - 考点一:函数极限的计算 - 3}

% --- 题目 ---
\problem[P4-7 (16-2,3)]{求极限 $\displaystyle \lim_{x \to 0} (\cos 2x + 2x \sin x)^{\frac{1}{x^4}}$.}
\ansat{239;【十年真题】 - 考点一:函数极限的计算 - 7}

% --- 题目 ---
\problem[P4-2 (25-2)]{$\displaystyle \lim_{n \to \infty} \frac{1}{n^2} \left[ \ln \frac{1}{n} + 2\ln \frac{2}{n} + \dots + (n-1)\ln \frac{n-1}{n} \right] = \underline{\hspace{3em}}.$}
\ansat{239;【十年真题】 - 考点二:数列极限的计算 - 2}

% --- 题目 ---
\problem[P4-5 (19-1,3)]{设 $\displaystyle a_n = \int_{0}^{1} x^n \sqrt{1 - x^2} \mathrm{d}x \ (n = 0, 1, 2, \dots)$.}
\begin{enumerate}
\item[(1)] 证明: 数列 $\displaystyle \{a_n\}$ 单调减少, 且 $\displaystyle a_n = \frac{n-1}{n+2} a_{n-2} \ (n=2, 3, \dots)$;
\item[(2)] 求 $\displaystyle \lim_{n \to \infty} \frac{a_n}{a_{n-1}}$.
\end{enumerate}
\ansat{239;【十年真题】 - 考点二:数列极限的计算 - 5}

% --- 题目 ---
\problem[P4-6 (18-1,2,3)]{设数列 $\displaystyle \{x_n\}$ 满足:$\displaystyle x_1 > 0, \ x_n \mathrm{e}^{x_{n+1}} = \mathrm{e}^{x_n} - 1 \ (n = 1, 2, \dots)$. 证明 $\displaystyle \{x_n\}$ 收敛, 并求 $\displaystyle \lim_{n \to \infty} x_n$.}
\ansat{240;【十年真题】 - 考点二:数列极限的计算 - 6}

% --- 题目 ---
\problem[P6-例1 (2)]{$\displaystyle \lim_{x \to 0} \frac{\sqrt{1+x} - \sqrt{1+\tan x}}{x \tan^2 x} = \underline{\hspace{3em}}.$}
\ansat{6;【方法探究】 - 考点一:函数极限的计算 - 例1 (2)}

% --- 题目 ---
\problem[P7-变式 1.1 (97-2)]{求极限 $\displaystyle \lim_{x \to -\infty} \frac{\sqrt{4x^2+x-1}+x+1}{\sqrt{x^2+\sin x}}$.}
\ansat{240;【方法探究】 - 考点一:函数极限的计算 - 变式 1.1}

% --- 题目 ---
\problem[P8-变式 3 (98-1)]{求极限 $$\lim_{n \to \infty} \left( \frac{\sin \frac{\pi}{n}}{n+1} + \frac{\sin \frac{2\pi}{n}}{n + \frac{1}{2}} + \dots + \frac{\sin \pi}{n + \frac{1}{n}} \right).$$}
\ansat{240;【方法探究】 - 考点二:数列极限的计算 - 变式 3}

% --- 题目 ---
\probtagged[P8-变式 4.1 (96-1)]{设 $\displaystyle x_1 = 10, \ x_{n+1} = \sqrt{6+x_n} \ (n = 1, 2, \dots)$. 试证数列 $\displaystyle \{x_n\}$ 极限存在, 并求此极限.}
\ansat{241;【方法探究】 - 考点二:数列极限的计算 - 变式 4.1}

% --- 题目 ---
\problem[P8-变式 4.2 (11-1, 2)]{(1) 证明:对任意的正整数 $\displaystyle n$, 都有 $\displaystyle \frac{1}{n+1} < \ln\left(1+\frac{1}{n}\right) < \frac{1}{n}$ 成立;
\newline
(2) 设 $\displaystyle a_n = 1 + \frac{1}{2} + \dots + \frac{1}{n} - \ln n (n = 1, 2, \dots)$, 证明数列 $\displaystyle \{a_n\}$ 收敛.}
\begin{note}
  主要错的是 (2)
\end{note}
\ansat{241;【方法探究】 - 考点二:数列极限的计算 - 变式 4.2}