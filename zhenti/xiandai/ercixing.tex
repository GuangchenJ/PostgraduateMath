% --- 题目 ---
\problem[P188-2 (25-2)]{设矩阵$\displaystyle \begin{pmatrix} 1 & 2 & 0 \\ 2 & a & 0 \\ 0 & 0 & b \end{pmatrix}$有一个正特征值和两个负特征值, 则( \quad )
\begin{tasks}(2)
  \task $a>4, b>0$.
  \task $a<4, b>0$.
  \task $a>4, b<0$.
  \task $a<4, b<0$.
\end{tasks}}
\ansat{347;【十年真题】 - 考点一:化二次型为标准型 - 2}

% --- 题目 ---
\problem[P189-12 (20-1,3)]{设二次型
$$f(x_1,x_2)=x_1^2-4x_1x_2+4x_2^2$$
经正交变换$\displaystyle \begin{pmatrix} x_1 \\ x_2 \end{pmatrix}=\bm{Q}\begin{pmatrix} y_1 \\ y_2 \end{pmatrix}$化为二次型
$$g(y_1,y_2)=ay_1^2+4y_1y_2+by_2^2,$$
其中$\displaystyle a \geqslant b$.
\begin{enumerate}
\item[(1)] 求$\displaystyle a, b$的值;
\item[(2)] 求正交矩阵$\displaystyle \bm{Q}$.
\end{enumerate}}
\ansat{348;【十年真题】 - 考点一:化二次型为标准型 - 12}

% --- 题目 ---
\problem[P189-2 (23-1)]{已知二次型
$$f(x_1,x_2,x_3)=x_1^2+2x_2^2+2x_3^2+2x_1x_2-2x_1x_3,$$
$$g(y_1,y_2,y_3)=y_1^2+y_2^2+y_3^2+2y_2y_3.$$
\begin{enumerate}
\item[(1)] 求可逆变换$\displaystyle \bm{x}=\bm{P}\bm{y}$将$\displaystyle f(x_1,x_2,x_3)$化为$\displaystyle g(y_1,y_2,y_3)$;
\item[(2)] 是否存在正交变换$\displaystyle \bm{x}=\bm{Q}\bm{y}$将$\displaystyle f(x_1,x_2,x_3)$化为$\displaystyle g(y_1,y_2,y_3)$?
\end{enumerate}}
\ansat{349;【十年真题】 - 考点二:矩阵的合同 - 2}

% --- 题目 ---
\problem[P190-1 (25-3)]{设矩阵$\displaystyle \bm{A}=\begin{pmatrix} 1 & 2 \\ -2 & -a \end{pmatrix}$, $\displaystyle \bm{B}=\begin{pmatrix} 1 & 0 \\ 1 & a \end{pmatrix}$. 若$\displaystyle f(x,y)=|x\bm{A}+y\bm{B}|$是正定二次型,则$\displaystyle a$的取值范围是( \quad )
\begin{tasks}(2)
  \task $(0,2-\sqrt{3})$.
  \task $(2-\sqrt{3},2+\sqrt{3})$.
  \task $(2+\sqrt{3},4)$.
  \task $(0,4)$.
\end{tasks}}
\ansat{350;【十年真题】 - 考点三:正定二次型与正定矩阵 - 1}
\vspace{10em}

% --- 题目 ---
\problem[P190-2 (21-1)]{设矩阵$\displaystyle \bm{A}=\begin{pmatrix} a & 1 & -1 \\ 1 & a & -1 \\ -1 & -1 & a \end{pmatrix}$.
\begin{enumerate}
\item[(1)] 求正交矩阵$\displaystyle \bm{P}$, 使$\displaystyle \bm{P}^{\mathrm{T}}\bm{A}\bm{P}$为对角矩阵;
\item[(2)] 求正定矩阵$\displaystyle \bm{C}$, 使$\displaystyle \bm{C}^2=(a+3)\bm{E}-\bm{A}$, 其中$\bm{E}$为3阶单位矩阵.
\end{enumerate}}
\begin{note}
  主要错的是 (2)
\end{note}
\ansat{350;【十年真题】 - 考点三:正定二次型与正定矩阵 - 2}

% --- 题目 ---
\problem[P193-3 (14-1,2,3)]{设二次型
$$f(x_1,x_2,x_3)=x_1^2-x_2^2+2ax_1x_3+4x_2x_3$$
的负惯性指数是1, 则$\displaystyle a$的取值范围是\underline{\hspace{4em}}.}
\ansat{351;【真题精选】 - 考点一:化二次型为标准型 - 3}

% --- 题目 ---
\problem[P193-8 (13-1,2,3)]{设二次型
$$f(x_1,x_2,x_3)=2 \left(a_1x_1+a_2x_2+a_3x_3\right)^2 + \left(b_1x_1+b_2x_2+b_3x_3\right)^2,$$
记$\displaystyle \bm{\alpha}=\begin{pmatrix} a_1 \\ a_2 \\ a_3 \end{pmatrix}$, $\displaystyle \bm{\beta}=\begin{pmatrix} b_1 \\ b_2 \\ b_3 \end{pmatrix}$.
\begin{enumerate}
\item[(1)] 证明二次型$\displaystyle f$对应的矩阵为$\displaystyle 2\bm{\alpha}\bm{\alpha}^{\mathrm{T}}+\bm{\beta}\bm{\beta}^{\mathrm{T}}$;
\item[(2)] 若$\displaystyle \bm{\alpha}, \bm{\beta}$正交且均为单位向量, 证明二次型$\displaystyle f$在正交变换下的标准形为$\displaystyle 2y_1^2+y_2^2$.
\end{enumerate}}
\begin{note}
  主要错的是 (2)
\end{note}
\ansat{352;【真题精选】 - 考点一:化二次型为标准型 - 8}

% --- 题目 ---
\problem[P193 (01-3)]{设$\displaystyle \bm{A}$为$\displaystyle n$阶实对称矩阵, $\displaystyle r\left(\bm{A}\right)=n$, $\displaystyle A_{ij}$是$\displaystyle \bm{A}=\left(a_{ij}\right)_{n \times n}$中元素$\displaystyle a_{ij}$的代数余子式 $\displaystyle (i,j=1,2,\cdots,n)$, 二次型$\displaystyle f(x_1,x_2,\cdots,x_n)=\sum_{i=1}^n \sum_{j=1}^n \frac{A_{ij}}{|\bm{A}|}x_i x_j$.
\begin{enumerate}
\item[(1)] 记$\displaystyle \bm{x}=(x_1,x_2,\cdots,x_n)^{\mathrm{T}}$, 把$\displaystyle f(x_1,x_2,\cdots,x_n)$写成矩阵形式, 并证明二次型$\displaystyle f(\bm{x})$的矩阵为$\displaystyle \bm{A}^{-1}$;
\item[(2)] 二次型$\displaystyle g(\bm{x})=\bm{x}^{\mathrm{T}}\bm{A}\bm{x}$与$\displaystyle f(\bm{x})$的规范形是否相同? 说明理由.
\end{enumerate}}
\ansat{352;【真题精选】 - 考点二:矩阵的合同}

% --- 题目 ---
\problem[P194-1 (05-3)]{设$\displaystyle \bm{D}=\begin{pmatrix} \bm{A} & \bm{C} \\ \bm{C}^{\mathrm{T}} & \bm{B} \end{pmatrix}$为正定矩阵, 其中$\displaystyle \bm{A}, \bm{B}$为$\displaystyle m$阶, $n$阶对称矩阵, $\displaystyle \bm{C}$为$\displaystyle m \times n$矩阵.
\begin{enumerate}
\item[(1)] 计算$\displaystyle \bm{P}^{\mathrm{T}}\bm{D}\bm{P}$, 其中$\displaystyle \bm{P}=\begin{pmatrix} \bm{E}_m & -\bm{A}^{-1}\bm{C} \\ \bm{O} & \bm{E}_n \end{pmatrix}$;
\item[(2)] 利用(1)的结果判断矩阵$\displaystyle \bm{B}-\bm{C}^{\mathrm{T}}\bm{A}^{-1}\bm{C}$是否为正定矩阵, 并证明你的结论.
\end{enumerate}}
\begin{note}
  主要错的是 (2)
\end{note}
\ansat{352;【真题精选】 - 考点三:正定二次型与正定矩阵 - 1}
