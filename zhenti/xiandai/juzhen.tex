% --- 题目 ---
\problem[P150-3 (24-1)]{设实矩阵 $\displaystyle \bm{A} = \begin{pmatrix} a+1 & a \\ a & a \end{pmatrix}$. 若对任意实向量 $\displaystyle \boldsymbol{\alpha} = \begin{pmatrix} x_1 \\ x_2 \end{pmatrix}, \ \boldsymbol{\beta} = \begin{pmatrix} y_1 \\ y_2 \end{pmatrix}$, 
$$\displaystyle (\boldsymbol{\alpha}^{\mathrm{T}} \bm{A} \boldsymbol{\beta})^2 \le \boldsymbol{\alpha}^{\mathrm{T}} \bm{A} \boldsymbol{\alpha} \cdot \boldsymbol{\beta}^{\mathrm{T}} \bm{A} \boldsymbol{\beta}$$ 
都成立, 则 $\displaystyle a$ 的取值范围为 \underline{\hspace{4em}}.}
\ansat{325;【十年真题】 - 考点一:矩阵的运算 - 3}

% --- 题目 ---
\problem[P150-4 (22-1)]{已知矩阵 $\displaystyle \bm{A}$ 和 $\displaystyle \bm{E}-\bm{A}$ 可逆, 其中 $\displaystyle \bm{E}$ 为单位矩阵. 若矩阵 $\displaystyle \bm{B}$ 满足 $\displaystyle [\bm{E} - (\bm{E} - \bm{A})^{-1}]\bm{B} = \bm{A}$, 则 $\displaystyle \bm{B} - \bm{A} = \underline{\hspace{4em}}. $}
\ansat{325;【十年真题】 - 考点一:矩阵的运算 - 4}

% --- 题目 ---
\problem[P150-1 (25-2)]{下列矩阵中, 可以经过若干初等行变换得到矩阵 $\displaystyle \begin{pmatrix} 1 & 1 & 0 & 1 \\ 0 & 0 & 1 & 2 \\ 0 & 0 & 0 & 0 \end{pmatrix}$ 的是 ( \quad )}
\begin{tasks}(2)
  \task $\displaystyle \begin{pmatrix} 1 & 1 & 0 & 1 \\ 1 & 2 & 1 & 3 \\ 2 & 3 & 1 & 4 \end{pmatrix}.$
  \task $\displaystyle \begin{pmatrix} 1 & 1 & 0 & 1 \\ 1 & 1 & 2 & 5 \\ 1 & 1 & 1 & 3 \end{pmatrix}.$
  \task $\displaystyle \begin{pmatrix} 1 & 0 & 0 & 1 \\ 0 & 1 & 0 & 3 \\ 0 & 1 & 0 & 0 \end{pmatrix}.$
  \task $\displaystyle \begin{pmatrix} 1 & 1 & 2 & 3 \\ 1 & 2 & 2 & 3 \\ 2 & 3 & 4 & 6 \end{pmatrix}.$
\end{tasks}
\ansat{325;【十年真题】 - 考点二:矩阵的初等变换与初等矩阵 - 1}

% --- 题目 ---
\problem[P150-4 (22-2,3)]{设 $\displaystyle \bm{A}$ 为 3 阶矩阵, 交换 $\displaystyle \bm{A}$ 的第 2 行和第 3 行, 再将第 2 列的 $\displaystyle -1$ 倍加到第 1 列, 得到 $\displaystyle \begin{pmatrix} -2 & 1 & -1 \\ 1 & -1 & 0 \\ -1 & 0 & 0 \end{pmatrix}$, 则 $\displaystyle \bm{A}^{-1}$ 的迹 $\displaystyle \mathrm{tr}(\bm{A}^{-1}) = \underline{\hspace{4em}}.$}
\ansat{325;【十年真题】 - 考点二:矩阵的初等变换与初等矩阵 - 4}

% --- 题目 ---
\problem[P150-2 (24-2)]{设 $\displaystyle \bm{A}$ 为 4 阶矩阵, $\displaystyle \bm{A}^*$ 为 $\displaystyle \bm{A}$ 的伴随矩阵. 若 $\displaystyle \bm{A}(\bm{A} - \bm{A}^*) = \bm{O}$, 且 $\displaystyle \bm{A} \neq \bm{A}^*$, 则 $\displaystyle r(\bm{A})$ 取值为 ( \quad )}
\begin{tasks}(2)
  \task $\displaystyle 0$ 或 $\displaystyle 1$.
  \task $\displaystyle 1$ 或 $\displaystyle 3$.
  \task $\displaystyle 2$ 或 $\displaystyle 3$.
  \task $\displaystyle 1$ 或 $\displaystyle 2$.
\end{tasks}
\ansat{325;【十年真题】 - 考点三:矩阵的秩与等价 - 2}

% --- 题目 ---
\problem[P156-10 (95-1)]{设 $\displaystyle \bm{A}$ 是 $\displaystyle n$ 阶矩阵, 满足 $\displaystyle \bm{A}\bm{A}^{\mathrm{T}}=\bm{E}$ ($\displaystyle \bm{E}$ 是 $\displaystyle n$ 阶单位矩阵, $\displaystyle \bm{A}^{\mathrm{T}}$ 是 $\displaystyle \bm{A}$ 的转置矩阵), $\displaystyle |\bm{A}| < 0$, 则 $\displaystyle |\bm{A} + \bm{E}| = \underline{\hspace{4em}}. $}
\ansat{327;【真题精选】 - 考点一:矩阵的运算 - 10}

% --- 题目 ---
\problem[P156-3 (01-3)]{设}
\[ \bm{A} = \begin{pmatrix} a_{11} & a_{12} & a_{13} & a_{14} \\ a_{21} & a_{22} & a_{23} & a_{24} \\ a_{31} & a_{32} & a_{33} & a_{34} \\ a_{41} & a_{42} & a_{43} & a_{44} \end{pmatrix}, \bm{B} = \begin{pmatrix} a_{14} & a_{13} & a_{12} & a_{11} \\ a_{24} & a_{23} & a_{22} & a_{21} \\ a_{34} & a_{33} & a_{32} & a_{31} \\ a_{44} & a_{43} & a_{42} & a_{41} \end{pmatrix}, \]
\[ \bm{P}_1 = \begin{pmatrix} 0 & 0 & 0 & 1 \\ 0 & 1 & 0 & 0 \\ 0 & 0 & 1 & 0 \\ 1 & 0 & 0 & 0 \end{pmatrix}, \bm{P}_2 = \begin{pmatrix} 1 & 0 & 0 & 0 \\ 0 & 0 & 1 & 0 \\ 0 & 1 & 0 & 0 \\ 0 & 0 & 0 & 1 \end{pmatrix}, \]
$\displaystyle \bm{A}$ 可逆, 则 $\displaystyle \bm{B}^{-1} = ( \quad )$
\begin{tasks}(2)
  \task $\displaystyle \bm{A}^{-1}\bm{P}_1\bm{P}_2.$
  \task $\displaystyle \bm{P}_1\bm{A}^{-1}\bm{P}_2.$
  \task $\displaystyle \bm{P}_1\bm{P}_2\bm{A}^{-1}.$
  \task $\displaystyle \bm{P}_2\bm{A}^{-1}\bm{P}_1.$
\end{tasks}
\ansat{327;【真题精选】 - 考点二:矩阵的初等变换与初等矩阵 - 3}

\vspace{9em}

% --- 题目 ---
\problem[P156-5 (08-1)]{设 $\displaystyle \boldsymbol{\alpha}, \boldsymbol{\beta}$ 为 3 维列向量, 矩阵 $\displaystyle \bm{A} = \boldsymbol{\alpha}\boldsymbol{\alpha}^{\mathrm{T}} + \boldsymbol{\beta}\boldsymbol{\beta}^{\mathrm{T}}$, 其中 $\displaystyle \boldsymbol{\alpha}^{\mathrm{T}}$ 为 $\displaystyle \boldsymbol{\alpha}$ 的转置, $\displaystyle \boldsymbol{\beta}^{\mathrm{T}}$ 为 $\displaystyle \boldsymbol{\beta}$ 的转置. 证明:}
\begin{enumerate}
\item[(1)] $\displaystyle r(\bm{A}) \le 2$;
\item[(2)] 若 $\displaystyle \boldsymbol{\alpha}, \boldsymbol{\beta}$ 线性相关, 则 $\displaystyle r(\bm{A}) < 2$.
\end{enumerate}
\ansat{327;【真题精选】 - 考点三:矩阵的秩与等价 - 5}
