% --- 题目 ---
\problem[P177-1 (24-1)]{设$\displaystyle \bm{A}$是秩为2的3阶矩阵, $\displaystyle \bm{\alpha}$是满足$\displaystyle \bm{A}\bm{\alpha}=\bm{0}$的非零向量. 若对满足$\displaystyle \bm{\beta}^{\mathrm{T}}\bm{\alpha}=0$的3维列向量$\displaystyle \bm{\beta}$,均有$\displaystyle \bm{A}\bm{\beta}=\bm{\beta}$,则( \quad )
\begin{tasks}(2)
  \task $\displaystyle \bm{A}^3$的迹为2.
  \task $\displaystyle \bm{A}^3$的迹为5.
  \task $\displaystyle \bm{A}^2$的迹为8.
  \task $\displaystyle \bm{A}^2$的迹为9.
\end{tasks}}
\ansat{340;【十年真题】 - 考点:矩阵的特征值和特征向量 - 1}

% --- 题目 ---
\problem[P177-3 (24-3)]{设$\displaystyle \bm{A}$为3阶矩阵, $\displaystyle bm{A}^*$为$\displaystyle \bm{A}$的伴随矩阵, $\displaystyle \bm{E}$为3阶单位矩阵. 若$\displaystyle r(2\bm{E}-\bm{A})=1$, $\displaystyle r(\bm{E}+\bm{A})=2$,则$\displaystyle |\bm{A}^*|=$\underline{\hspace{4em}}.}
\ansat{340;【十年真题】 - 考点:矩阵的特征值和特征向量 - 3}

% --- 题目 ---
\problem[P177-5 (18-1)]{设2阶矩阵$\displaystyle \bm{A}$有两个不同特征值, $\displaystyle \bm{\alpha}_1,\bm{\alpha}_2$ 是 $\displaystyle \bm{A}$的线性无关的特征向量, 且满足
$$\bm{A}^2(\bm{\alpha}_1 + \bm{\alpha}_2) = \bm{\alpha}_1 + \bm{\alpha}_2,$$
则$\displaystyle |\bm{A}|=$\underline{\hspace{4em}}.}
\ansat{340;【十年真题】 - 考点:矩阵的特征值和特征向量 - 5}

% --- 题目 ---
\problem[P179-变式 1.2]{设3阶实对称矩阵$\displaystyle \bm{A}$的秩为2,$\displaystyle \bm{A}^2=\bm{A}$,且$\bm{A}(1,-1,1)^{\mathrm{T}}=\bm{0}$,求$\bm{A}$的特征值与特征向量.}
\ansat{341;【方法探究】 - 考点:矩阵的特征值和特征向量 - 变式 1.2}

% --- 题目 ---
\problem[P179-1 (08-1,2,3)]{设$\displaystyle \bm{A}$为$\displaystyle n$阶非零矩阵, $\displaystyle \bm{E}$为$\displaystyle n$阶单位矩阵. 若$\displaystyle \bm{A}^3=\bm{O}$,则( \quad )
\begin{tasks}(2)
  \task $\displaystyle \bm{E}-\bm{A}$不可逆, $\displaystyle \bm{E}+\bm{A}$不可逆.
  \task $\displaystyle \bm{E}-\bm{A}$不可逆, $\displaystyle \bm{E}+\bm{A}$可逆.
  \task $\displaystyle \bm{E}-\bm{A}$可逆, $\displaystyle \bm{E}+\bm{A}$可逆.
  \task $\displaystyle \bm{E}-\bm{A}$可逆, $\displaystyle \bm{E}+\bm{A}$不可逆.
\end{tasks}}
\ansat{341;【真题精选】 - 考点:矩阵的特征值和特征向量 - 1}

% --- 题目 ---
\problem[P179-3 (02-3)]{设$\displaystyle \bm{A}$是$\displaystyle n$阶实对称矩阵, $\displaystyle \bm{P}$是$\displaystyle n$阶可逆矩阵, 已知$\displaystyle n$维列向量$\displaystyle \bm{\alpha}$是$\displaystyle \bm{A}$的属于特征值$\displaystyle \lambda$的特征向量, 则矩阵$\displaystyle \left(\bm{P}^{-1}\bm{A}\bm{P}\right)^{\mathrm{T}}$属于特征值$\displaystyle \lambda$的特征向量是( \quad )
\begin{tasks}(2)
  \task $\displaystyle \bm{P}^{-1}\bm{\alpha}$.
  \task $\displaystyle \bm{P}^{\mathrm{T}}\bm{\alpha}$.
  \task $\displaystyle \bm{P}\bm{\alpha}$.
  \task $\displaystyle \left(\bm{P}^{-1}\right)^{\mathrm{T}}\bm{\alpha}$.
\end{tasks}}
\ansat{341;【真题精选】 - 考点:矩阵的特征值和特征向量 - 3}

% --- 题目 ---
\problem[P179-36 (96-1)]{设$\displaystyle \bm{A}=\bm{E}-\bm{\xi}\bm{\xi}^{\mathrm{T}}$, 其中$\displaystyle \bm{E}$是$\displaystyle n$阶单位矩阵, $\displaystyle \bm{\xi}$是$n$维非零列向量, $\displaystyle \bm{\xi}^{\mathrm{T}}$是$\displaystyle \bm{\xi}$的转置. 证明:
\begin{enumerate}
\item[(1)] $\displaystyle \bm{A}^2=\bm{A}$的充分条件是$\displaystyle \bm{\xi}^{\mathrm{T}}\bm{\xi}=1$
\item[(2)] 当$\displaystyle \bm{\xi}^{\mathrm{T}}\bm{\xi}=1$时, $\displaystyle \bm{A}$是不可逆矩阵.
\end{enumerate}}
\ansat{341;【真题精选】 - 考点:矩阵的特征值和特征向量 - 6}

% --- 题目 ---
\problem[P180-2 (24-2)]{设$\displaystyle \bm{A},\bm{B}$为2阶矩阵, 且$\displaystyle \bm{AB}=\bm{BA}$, 则“$\displaystyle \bm{A}$有两个不相等的特征值”是“$\displaystyle \bm{B}$可对角化”的( \quad )
\begin{tasks}(2)
  \task 充分必要条件.
  \task 充分不必要条件.
  \task 必要不充分条件.
  \task 既不充分也不必要条件.
\end{tasks}}
\ansat{341;【十年真题】 - 考点:矩阵的相似和相似对角化 - 2}

% --- 题目 ---
\problem[P180-4 (22-1)]{下述四个条件中, 3阶矩阵$\displaystyle \bm{A}$可对角化的一个充分但不必要条件是( \quad )
\begin{tasks}(2)
  \task $\displaystyle \bm{A}$有3个互不相等的特征值.
  \task $\displaystyle \bm{A}$有3个线性无关的特征向量.
  \task $\displaystyle \bm{A}$有3个两两线性无关的特征向量.
  \task $\displaystyle \bm{A}$的属于不同特征值的特征向量正交.
\end{tasks}}
\ansat{341;【十年真题】 - 考点:矩阵的相似和相似对角化 - 4}


% --- 题目 ---
\problem[P180-5 (22-2,3)]{设$\displaystyle \bm{A}$为3阶矩阵, $\displaystyle \bm{\Lambda}=\begin{pmatrix} 1 & 0 & 0 \\ 0 & -1 & 0 \\ 0 & 0 & 0 \end{pmatrix}$, 则$\displaystyle \bm{A}$的特征值为 $\displaystyle 1,-1,0$ 的充分必要条件是( \quad )
\begin{tasks}(2)
  \task 存在可逆矩阵$\displaystyle \bm{P},\bm{Q}$, 使得$\displaystyle \bm{A}=\bm{P}\bm{\Lambda}\bm{Q}$.
  \task 存在可逆矩阵$\displaystyle \bm{P}$, 使得$\displaystyle  \bm{A}=\bm{P}\bm{\Lambda}\bm{P}^{-1}$.
  \task 存在正交矩阵$\displaystyle \bm{Q}$, 使得$\displaystyle \bm{A}=\bm{Q}\bm{\Lambda}\bm{Q}^{-1}$.
  \task 存在可逆矩阵$\displaystyle \bm{P}$, 使得$\displaystyle \bm{A}=\bm{P}\bm{\Lambda}\bm{P}^{\mathrm{T}}$.
\end{tasks}}
\ansat{341;【十年真题】 - 考点:矩阵的相似和相似对角化 - 5}

% --- 题目 ---
\problem[P180-8 (17-1,2,3)]{已知矩阵$\displaystyle \bm{A}=\begin{pmatrix} 2 & 0 & 0 \\ 0 & 2 & 1 \\ 0 & 0 & 1 \end{pmatrix}$, $\displaystyle \bm{B}=\begin{pmatrix} 2 & 1 & 0 \\ 0 & 2 & 0 \\ 0 & 0 & 1 \end{pmatrix}$, $\displaystyle \bm{C}=\begin{pmatrix} 1 & 0 & 0 \\ 0 & 2 & 0 \\ 0 & 0 & 2 \end{pmatrix}$,则( \quad )
\begin{tasks}(2)
  \task $\bm{A}$与$\bm{C}$相似, $\bm{B}$与$\bm{C}$相似.
  \task $\bm{A}$与$\bm{C}$相似, $\bm{B}$与$\bm{C}$不相似.
  \task $\bm{A}$与$\bm{C}$不相似, $\bm{B}$与$\bm{C}$相似.
  \task $\bm{A}$与$\bm{C}$不相似, $\bm{B}$与$\bm{C}$不相似.
\end{tasks}}
\ansat{342;【十年真题】 - 考点:矩阵的相似和相似对角化 - 8}

% --- 题目 ---
\problem[P181-11 (18-2)]{设$\displaystyle \bm{A}$为3阶矩阵, $\displaystyle \bm{\alpha}_1,\bm{\alpha}_2,\bm{\alpha}_3$为线性无关的向量组. 若$\displaystyle \bm{A}\bm{\alpha}_1=2\bm{\alpha}_1+\bm{\alpha}_2+\bm{\alpha}_3$, $\displaystyle \bm{A}\bm{\alpha}_2=\bm{\alpha}_2+2\bm{\alpha}_3$, $\displaystyle \bm{A}\bm{\alpha}_3=-\bm{\alpha}_2+\bm{\alpha}_3$,则$\bm{A}$的实特征值为\underline{\hspace{4em}}.}
\ansat{342;【十年真题】 - 考点:矩阵的相似和相似对角化 - 11}

% --- 题目 ---
\problem[P181-15 (20-1,2,3)]{设$\displaystyle \bm{A}$为2阶矩阵, $\displaystyle \bm{P}=\left(\bm{\alpha},\bm{A}\bm{\alpha}\right)$, 其中$\displaystyle \bm{\alpha}$是非零向量且不是$\displaystyle \bm{A}$的特征向量.
\begin{enumerate}
\item[(1)] 证明$\displaystyle \bm{P}$为可逆矩阵;
\item[(2)] 若$\displaystyle \bm{A}^2\bm{\alpha}+\bm{A}\bm{\alpha}-6\bm{\alpha}=\bm{0}$, 求$\displaystyle \bm{P}^{-1}\bm{A}\bm{P}$, 并判断$\displaystyle \bm{A}$是否相似于对角矩阵.
\end{enumerate}}
\ansat{343;【十年真题】 - 考点:矩阵的相似和相似对角化 - 15}

% --- 题目 ---
\problem[P181-17 (16-1,2,3)]{已知矩阵$\displaystyle \bm{A}=\begin{pmatrix} 0 & -1 & 1 \\ 2 & -3 & 0 \\ 0 & 0 & 0 \end{pmatrix}$.
\begin{enumerate}
\item[(1)] 求$\displaystyle \bm{A}^{99}$;
\item[(2)] 设3阶矩阵$\displaystyle \bm{B}=(\bm{\alpha}_1,\bm{\alpha}_2,\bm{\alpha}_3)$满足$\displaystyle \bm{B}^2=\bm{BA}$. 记$\displaystyle \bm{B}^{100}=\left(\bm{\beta}_1,\bm{\beta}_2,\bm{\beta}_3\right)$, 将$\displaystyle \bm{\beta}_1,\bm{\beta}_2,\bm{\beta}_3$分别表示为$\displaystyle \bm{\alpha}_1,\bm{\alpha}_2,\bm{\alpha}_3$的线性组合.
\end{enumerate}}
\ansat{343;【十年真题】 - 考点:矩阵的相似和相似对角化 - 17}

% --- 题目 ---
\problem[P184-变式 2 (11-1,2,3)]{设$\displaystyle \bm{A}$为3阶实对称矩阵, $\displaystyle \bm{A}$的秩为2,且
$$\bm{A}\begin{pmatrix} 1 & 1 \\ 0 & 0 \\ -1 & 1 \end{pmatrix} = \begin{pmatrix} -1 & 1 \\ 0 & 0 \\ 1 & 1 \end{pmatrix}.$$
\begin{enumerate}
\item[(1)] 求$\displaystyle \bm{A}$的所有特征值与特征向量;
\item[(2)] 求矩阵$\displaystyle \bm{A}$.
\end{enumerate}}
\ansat{344;【方法探究】 - 考点:矩阵的相似和相似对角化 - 变式2}
\vspace{10em}

% --- 题目 ---
\problem[P185-6 (08-2,3)]{设$\displaystyle \bm{A}$为3阶矩阵, $\displaystyle \bm{\alpha}_1,\bm{\alpha}_2$为$\displaystyle \bm{A}$的分别属于特征值$\displaystyle -1,1$的特征向量, 向量$\displaystyle \bm{\alpha}_3$满足$\displaystyle \bm{A}\bm{\alpha}_3=\bm{\alpha}_2+\bm{\alpha}_3$.
\begin{enumerate}
\item[(1)] 证明$\displaystyle \bm{\alpha}_1,\bm{\alpha}_2,\bm{\alpha}_3$线性无关;
\item[(2)] 令$\displaystyle \bm{P} = \left(\bm{\alpha}_1,\bm{\alpha}_2,\bm{\alpha}_3\right)$,求$\displaystyle \bm{P}^{-1}\bm{A}\bm{P}$.
\end{enumerate}}
\ansat{345;【真题精选】 - 考点:矩阵的相似和相似对角化 - 6}

% --- 题目 ---
\problem[P185-7 (07-1,2,3)]{设3阶实对称矩阵$\displaystyle \bm{A}$的特征值$\lambda_1=1$, $\lambda_2=2$, $\lambda_3=-2$,且$\displaystyle \bm{\alpha}_1=(1,-1,1)^{\mathrm{T}}$是$\displaystyle \bm{A}$的属于$\displaystyle \lambda_1$的一个特征向量. 记$\displaystyle \bm{B}=\bm{A}^5-4\bm{A}^3+\bm{E}$, 其中$\bm{E}$为3阶单位矩阵.
\begin{enumerate}
\item[(1)] 验证$\displaystyle \bm{\alpha}_1$是矩阵$\displaystyle \bm{B}$的特征向量, 并求$\displaystyle \bm{B}$的全部特征值与特征向量;
\item[(2)] 求矩阵 $\displaystyle \bm{B}$ .
\end{enumerate}}
\ansat{345;【真题精选】 - 考点:矩阵的相似和相似对角化 - 7}

% --- 题目 ---
\problem[P185-9 (04-1,2)]{设矩阵$\displaystyle \bm{A}=\begin{pmatrix} 1 & 2 & -3 \\ -1 & 4 & -3 \\ 1 & a & 5 \end{pmatrix}$的特征方程有一个二重根, 求$\displaystyle a$的值, 并讨论$\displaystyle \bm{A}$是否可相似对角化.}
\ansat{345;【真题精选】 - 考点:矩阵的相似和相似对角化 - 9}

% --- 题目 ---
\problem[P186-10 (02-1)]{设$\displaystyle \bm{A}, \bm{B}$为同阶方阵.
\begin{enumerate}
\item[(1)] 如果$\displaystyle \bm{A}, \bm{B}$相似, 试证$\displaystyle \bm{A}, \bm{B}$的特征多项式相等;
\end{enumerate}}
\ansat{345;【真题精选】 - 考点:矩阵的相似和相似对角化 - 10}
\vspace{12em}

% --- 题目 ---
\problem[P186-12 (00-1)]{某试验性生产线每年1月份进行熟练工与非熟练工的人数统计, 然后将$\displaystyle \frac{1}{6}$熟练工支援其他生产部门, 其缺额由招收新的非熟练工补齐. 新、老非熟练工经过培训及实践至年终考核有$\displaystyle \frac{2}{5}$成为熟练工. 设第$\displaystyle n$年1月份统计的熟练工和非熟练工所占百分比分别为$\displaystyle x_n$和$\displaystyle y_n$, 记成向量$\displaystyle \begin{pmatrix} x_n \\ y_n \end{pmatrix}$.
\begin{enumerate}
\item[(1)] 求$\displaystyle \begin{pmatrix} x_{n+1} \\ y_{n+1} \end{pmatrix}$与$\begin{pmatrix} x_n \\ y_n \end{pmatrix}$的关系式并写成矩阵形式:
$$\begin{pmatrix} x_{n+1} \\ y_{n+1} \end{pmatrix} = \bm{A}\begin{pmatrix} x_n \\ y_n \end{pmatrix};$$
\item[(2)] 验证$\displaystyle \bm{\eta}_1=\begin{pmatrix} 4 \\ 1 \end{pmatrix}$, $\displaystyle \bm{\eta}_2=\begin{pmatrix} -1 \\ 1 \end{pmatrix}$是$\bm{A}$的两个线性无关的特征向量,并求出相应的特征值;
\item[(3)] 当$\displaystyle \begin{pmatrix} x_1 \\ y_1 \end{pmatrix} = \begin{pmatrix} \frac{1}{2} \\ \frac{1}{2} \end{pmatrix}$时, 求$\displaystyle \begin{pmatrix} x_{n+1} \\ y_{n+1} \end{pmatrix}$.
\end{enumerate}}
\ansat{346;【真题精选】 - 考点:矩阵的相似和相似对角化 - 12}

% --- 题目 ---
\problem[P187-15 (92-4)]{设矩阵$\displaystyle \bm{A}, \bm{B}$相似, 其中$\displaystyle \bm{A}=\begin{pmatrix} -2 & 0 & 0 \\ 2 & x & 2 \\ 3 & 1 & 1 \end{pmatrix}$, $\displaystyle \bm{B}=\begin{pmatrix} -1 & 0 & 0 \\ 0 & 2 & 0 \\ 0 & 0 & y \end{pmatrix}$.
\begin{enumerate}
\item[(1)] 求$\displaystyle x, y$的值;
\item[(2)] 求可逆矩阵$\displaystyle \bm{P}$, 使$\displaystyle \bm{P}^{-1}\bm{A}\bm{P}=\bm{B}$.
\end{enumerate}}
\ansat{346;【真题精选】 - 考点:矩阵的相似和相似对角化 - 12}